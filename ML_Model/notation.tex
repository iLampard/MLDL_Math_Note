% \chapter*{Notation}
% \label{notation}
\chapter*{符号声明}
\label{notation}


\typeout{START_CHAPTER "notation" \theabspage}

% Sometimes we have to include the following line to get this section
% included in the Table of Contents despite being a chapter*
% \addcontentsline{toc}{chapter}{Notation}
% This section provides a concise reference describing notation used throughout this
% document.
% If you are unfamiliar with any of the corresponding mathematical concepts,
% \citet{dlbook} describe most of these ideas in
% chapters 2--4.
\addcontentsline{toc}{chapter}{Notation}
本节提供了对整个文档中使用的数学符号的简要参考。
如果你对其中任何相应的数学概念不熟悉,大部分内容可参照\citet{dlbook}的2-4章节中的描述。

\vspace{\notationgap}
% Need to use minipage to keep title of table on same page as table
\begin{minipage}{\textwidth}
% This is a hack to put a little title over the table
% We cannot use "\section*", etc., they appear in the table of contents.
% tocdepth does not work on this chapter.
% \centerline{\bf Numbers and Arrays}
\centerline{\bf 数字和数组}
\bgroup
% The \arraystretch definition here increases the space between rows in the table,
% so that \displaystyle math has more vertical space.
\def\arraystretch{1.5}
\begin{tabular}{cp{3.25in}}
% $\displaystyle a$ & A scalar (integer or real)\\
% $\displaystyle \va$ & A vector\\
% $\displaystyle \mA$ & A matrix\\
% $\displaystyle \tA$ & A tensor\\
% $\displaystyle \mI_n$ & Identity matrix with $n$ rows and $n$ columns\\
% $\displaystyle \mI$ & Identity matrix with dimensionality implied by context\\
% $\displaystyle \ve^{(i)}$ & Standard basis vector $[0,\dots,0,1,0,\dots,0]$ with a 1 at position $i$\\
% $\displaystyle \text{diag}(\va)$ & A square, diagonal matrix with diagonal entries given by $\va$\\
% $\displaystyle \ra$ & A scalar random variable\\
% $\displaystyle \rva$ & A vector-valued random variable\\
% $\displaystyle \rmA$ & A matrix-valued random variable\\
$\displaystyle a$ & 标量(整数或实数)\\
$\displaystyle \va$ & 矢量\\
$\displaystyle \mA$ & 矩阵\\
$\displaystyle \tA$ & 张量\\
$\displaystyle \mI_n$ &  $n$ 行 $n$ 列的单位矩阵\\
$\displaystyle \mI$ & 上下文所指示维度的单位矩阵\\
$\displaystyle \ve^{(i)}$ & 标准基向量 $[0,\dots,0,1,0,\dots,0]$ ,其中第 $i$ 位为 1\\
$\displaystyle \text{diag}(\va)$ & 正方形对角矩阵,其中对角线元素为 $\va$ \\
$\displaystyle \ra$ & 标量随机变量\\
$\displaystyle \rva$ & 矢量随机变量\\
$\displaystyle \rmA$ & 矩阵随机变量\\
\end{tabular}
\egroup
\index{Scalar}
\index{Vector}
\index{Matrix}
\index{Tensor}
\end{minipage}

\vspace{\notationgap}
\begin{minipage}{\textwidth}
% \centerline{\bf Sets and Graphs}
\centerline{\bf 集合和图象}
\bgroup
\def\arraystretch{1.5}
\begin{tabular}{cp{3.25in}}
% $\displaystyle \sA$ & A set\\
% $\displaystyle \R$ & The set of real numbers \\
$\displaystyle \sA$ & 集合\\
$\displaystyle \R$ & 实数集合 \\
% NOTE: do not use \R^+, because it is ambiguous whether:
% - It includes 0
% - It includes only real numbers, or also infinity.
% We usually do not include infinity, so we may explicitly write
% [0, \infty) to include 0
% (0, \infty) to not include 0
% $\displaystyle \{0, 1\}$ & The set containing 0 and 1 \\
% $\displaystyle \{0, 1, \dots, n \}$ & The set of all integers between $0$ and $n$\\
% $\displaystyle [a, b]$ & The real interval including $a$ and $b$\\
% $\displaystyle (a, b]$ & The real interval excluding $a$ but including $b$\\
% $\displaystyle \sA \backslash \sB$ & Set subtraction, i.e., the set containing the elements of $\sA$ that are not in $\sB$\\
% $\displaystyle \gG$ & A graph\\
% $\displaystyle \parents_\gG(\ervx_i)$ & The parents of $\ervx_i$ in $\gG$
$\displaystyle \{0, 1\}$ & 包含 0 和 1 的集合 \\
$\displaystyle \{0, 1, \dots, n \}$ & 从 $0$ 到 $n$ 的所有整数的集合 \\
$\displaystyle [a, b]$ & 包含 $a$ 和 $b$ 的实数区间\\
$\displaystyle (a, b]$ & 不包含 $a$ 但包含 $b$ 的实数区间\\
$\displaystyle \sA \backslash \sB$ & 差集, 即集合包含了在 $\sA$ 中但不在 $\sB$ 中的元素\\
$\displaystyle \gG$ & 图象\\
$\displaystyle \parents_\gG(\ervx_i)$ & $\gG$ 中 $\ervx_i$ 的母元素
\end{tabular}
\egroup
\index{Scalar}
\index{Vector}
\index{Matrix}
\index{Tensor}
\index{Graph}
\index{Set}
\end{minipage}

\vspace{\notationgap}
\begin{minipage}{\textwidth}
% \centerline{\bf Indexing}
\centerline{\bf 角标}
\bgroup
\def\arraystretch{1.5}
\begin{tabular}{cp{3.25in}}
% $\displaystyle \eva_i$ & Element $i$ of vector $\va$, with indexing starting at 1 \\
% $\displaystyle \eva_{-i}$ & All elements of vector $\va$ except for element $i$ \\
% $\displaystyle \emA_{i,j}$ & Element $i, j$ of matrix $\mA$ \\
% $\displaystyle \mA_{i, :}$ & Row $i$ of matrix $\mA$ \\
% $\displaystyle \mA_{:, i}$ & Column $i$ of matrix $\mA$ \\
% $\displaystyle \etA_{i, j, k}$ & Element $(i, j, k)$ of a 3-D tensor $\tA$\\
% $\displaystyle \tA_{:, :, i}$ & 2-D slice of a 3-D tensor\\
% $\displaystyle \erva_i$ & Element $i$ of the random vector $\rva$ \\
$\displaystyle \eva_i$ & 矢量 $\va$ 的第 $i$ 个元素, 其中角标从 1 开始 \\
$\displaystyle \eva_{-i}$ & 矢量 $\va$ 除 $i$ 以外的所有元素 \\
$\displaystyle \emA_{i,j}$ & 矩阵 $\mA$ 的 第 $i$ 行第 $j$ 列元素\\
$\displaystyle \mA_{i, :}$ & 矩阵 $\mA$ 的第 $i$ 行\\
$\displaystyle \mA_{:, i}$ & 矩阵 $\mA$ 的第 $i$ 列 \\
$\displaystyle \etA_{i, j, k}$ & 3-D 张量 $\tA$ 的元素 $(i, j, k)$\\
$\displaystyle \tA_{:, :, i}$ & 3-D 张量的 2-D 切片\\
$\displaystyle \erva_i$ & 随机矢量 $\rva$ 的第 $i$ 个元素\\
\end{tabular}
\egroup
\end{minipage}

\vspace{\notationgap}
\begin{minipage}{\textwidth}
% \centerline{\bf Linear Algebra Operations}
\centerline{\bf 线性代数运算}
\bgroup
\def\arraystretch{1.5}
\begin{tabular}{cp{3.25in}}
% $\displaystyle \mA^\top$ & Transpose of matrix $\mA$ \\
% $\displaystyle \mA^+$ & Moore-Penrose pseudoinverse of $\mA$\\
% $\displaystyle \mA \odot \mB $ & Element-wise (Hadamard) product of $\mA$ and $\mB$ \\
% Wikipedia uses \circ for element-wise multiplication but this could be confused with function composition
% $\displaystyle \mathrm{det}(\mA)$ & Determinant of $\mA$ \\
$\displaystyle \mA^\top$ & 矩阵 $\mA$ 的转置 \\
$\displaystyle \mA^+$ & 矩阵 $\mA$ 的摩尔彭罗斯伪逆(广义逆)\\
$\displaystyle \mA \odot \mB $ & 矩阵 $\mA$ 和 $\mB$ 的元素积(Hadamard乘积)\\
$\displaystyle \mathrm{det}(\mA)$ & 矩阵 $\mA$ 的行列式\\
\end{tabular}
\egroup
\index{Transpose}
\index{Element-wise product|see {Hadamard product}}
\index{Hadamard product}
\index{Determinant}
\end{minipage}

\vspace{\notationgap}
\begin{minipage}{\textwidth}
% \centerline{\bf Calculus}
\centerline{\bf 微积分}
\bgroup
\def\arraystretch{1.5}
\begin{tabular}{cp{3.25in}}
% NOTE: the [2ex] on the next line adds extra height to that row of the table.
% Without that command, the fraction on the first line is too tall and collides
% with the fraction on the second line.
% $\displaystyle\frac{d y} {d x}$ & Derivative of $y$ with respect to $x$\\ [2ex]
% $\displaystyle \frac{\partial y} {\partial x} $ & Partial derivative of $y$ with respect to $x$ \\
% $\displaystyle \nabla_\vx y $ & Gradient of $y$ with respect to $\vx$ \\
% $\displaystyle \nabla_\mX y $ & Matrix derivatives of $y$ with respect to $\mX$ \\
% $\displaystyle \nabla_\tX y $ & Tensor containing derivatives of $y$ with respect to $\tX$ \\
% $\displaystyle \frac{\partial f}{\partial \vx} $ & Jacobian matrix $\mJ \in \R^{m\times n}$ of $f: \R^n \rightarrow \R^m$\\
% $\displaystyle \nabla_\vx^2 f(\vx)\text{ or }\mH( f)(\vx)$ & The Hessian matrix of $f$ at input point $\vx$\\
% $\displaystyle \int f(\vx) d\vx $ & Definite integral over the entire domain of $\vx$ \\
% $\displaystyle \int_\sS f(\vx) d\vx$ & Definite integral with respect to $\vx$ over the set $\sS$ \\
$\displaystyle\frac{d y} {d x}$ & $y$ 关于 $x$ 的导数\\ [2ex]
$\displaystyle \frac{\partial y} {\partial x} $ & $y$ 关于 $x$ 的偏导数\\
$\displaystyle \nabla_\vx y $ & $y$ 关于 $\vx$ 的梯度\\
$\displaystyle \nabla_\mX y $ & $y$ 关于 $\mX$ 的矩阵导数\\
$\displaystyle \nabla_\tX y $ & $y$ 关于 $\tX$ 的张量导数\\
$\displaystyle \frac{\partial f}{\partial \vx} $ & $f: \R^n \rightarrow \R^m$ 的雅克比矩阵 $\mJ \in \R^{m\times n}$\\
$\displaystyle \nabla_\vx^2 f(\vx)\text{ or }\mH( f)(\vx)$ & $f$ 在输入点 $\vx$ 的海森矩阵\\
$\displaystyle \int f(\vx) d\vx $ & 在全域上关于 $\vx$ 的定积分\\
$\displaystyle \int_\sS f(\vx) d\vx$ & 在集合 $\sS$ 上关于 $\vx$ 的定积分\\
\end{tabular}
\egroup
\index{Derivative}
\index{Integral}
\index{Jacobian matrix}
\index{Hessian matrix}
\end{minipage}

\vspace{\notationgap}
\begin{minipage}{\textwidth}
% \centerline{\bf Probability and Information Theory}
\centerline{\bf 概率论与信息论}
\bgroup
\def\arraystretch{1.5}
\begin{tabular}{cp{3.25in}}
% $\displaystyle \ra \bot \rb$ & The random variables $\ra$ and $\rb$ are independent\\
% $\displaystyle \ra \bot \rb \mid \rc $ & They are conditionally independent given $\rc$\\
% $\displaystyle P(\ra)$ & A probability distribution over a discrete variable\\
% $\displaystyle p(\ra)$ & A probability distribution over a continuous variable, or over
% a variable whose type has not been specified\\
% $\displaystyle \ra \sim P$ & Random variable $\ra$ has distribution $P$\\% so thing on left of \sim should always be a random variable, with name beginning with \r
% $\displaystyle  \E_{\rx\sim P} [ f(x) ]\text{ or } \E f(x)$ & Expectation of $f(x)$ with respect to $P(\rx)$ \\
% $\displaystyle \Var(f(x)) $ &  Variance of $f(x)$ under $P(\rx)$ \\
% $\displaystyle \Cov(f(x),g(x)) $ & Covariance of $f(x)$ and $g(x)$ under $P(\rx)$\\
% $\displaystyle H(\rx) $ & Shannon entropy of the random variable $\rx$\\
% $\displaystyle \KL ( P \Vert Q ) $ & Kullback-Leibler divergence of P and Q \\
% $\displaystyle \mathcal{N} ( \vx ; \vmu , \mSigma)$ & Gaussian distribution %
% over $\vx$ with mean $\vmu$ and covariance $\mSigma$ \\
$\displaystyle \ra \bot \rb$ & 随机变量 $\ra$ 与 $\rb$ 相互独立\\
$\displaystyle \ra \bot \rb \mid \rc $ & 对于给定的 $\rc$ 条件独立\\
$\displaystyle P(\ra)$ & 离散变量的概率分布\\
$\displaystyle p(\ra)$ & 连续变量的概率分布或类型不确定的变量的概率分布 \\
$\displaystyle \ra \sim P$ & 随机变量 $\ra$ 服从 $P$ 分布\\% so thing on left of \sim should always be a random variable, with name beginning with \r
$\displaystyle  \E_{\rx\sim P} [ f(x) ]\text{ or } \E f(x)$ & $f(x)$ 关于 $P(\rx)$ 的期望\\
$\displaystyle \Var(f(x)) $ & $f(x)$ 在 $P(\rx)$ 下的方差\\
$\displaystyle \Cov(f(x),g(x)) $ & $f(x)$ 和 $g(x)$ 在 $P(\rx)$ 下的协方差\\
$\displaystyle H(\rx) $ & 随机变量 $\rx$ 的信息熵 \\
$\displaystyle \KL ( P \Vert Q ) $ & P 和 Q 的相对熵 \\
$\displaystyle \mathcal{N} ( \vx ; \vmu , \mSigma)$ & 均值为 $\vmu$ 方差为 $\mSigma$ 的 $\vx$ 的高斯分布 \\
\end{tabular}
\egroup
\index{Independence}
\index{Conditional independence}
\index{Variance}
\index{Covariance}
\index{Kullback-Leibler divergence}
\index{Shannon entropy}
\end{minipage}

\vspace{\notationgap}
\begin{minipage}{\textwidth}
% \centerline{\bf Functions}
\centerline{\bf 函数}
\bgroup
\def\arraystretch{1.5}
\begin{tabular}{cp{3.25in}}
% $\displaystyle f: \sA \rightarrow \sB$ & The function $f$ with domain $\sA$ and range $\sB$\\
% $\displaystyle f \circ g $ & Composition of the functions $f$ and $g$ \\
%   $\displaystyle f(\vx ; \vtheta) $ & A function of $\vx$ parametrized by $\vtheta$.
%   (Sometimes we write $f(\vx)$ and omit the argument $\vtheta$ to lighten notation) \\
% $\displaystyle \log x$ & Natural logarithm of $x$ \\
% $\displaystyle \sigma(x)$ & Logistic sigmoid, $\displaystyle \frac{1} {1 + \exp(-x)}$ \\
% $\displaystyle \zeta(x)$ & Softplus, $\log(1 + \exp(x))$ \\
% $\displaystyle || \vx ||_p $ & $\normlp$ norm of $\vx$ \\
% $\displaystyle || \vx || $ & $\normltwo$ norm of $\vx$ \\
% $\displaystyle x^+$ & Positive part of $x$, i.e., $\max(0,x)$\\
% $\displaystyle \1_\mathrm{condition}$ & is 1 if the condition is true, 0 otherwise\\
$\displaystyle f: \sA \rightarrow \sB$ & 定义域为 $\sA$ 值域为 $\sB$ 的函数 $f$ \\
$\displaystyle f \circ g $ & 函数 $f$ 和 $g$ 的合成 \\
  $\displaystyle f(\vx ; \vtheta) $ & 参数为 $\vtheta$ 的 $\vx$ 的函数.
  (有时我们写作 $f(\vx)$ 而忽略参数 $\vtheta$ 来简化符号)\\
$\displaystyle \log x$ & $x$ 的自然对数 \\
$\displaystyle \sigma(x)$ & Logistic sigmoid 函数, $\displaystyle \frac{1} {1 + \exp(-x)}$ \\
$\displaystyle \zeta(x)$ & Softplus 函数, $\log(1 + \exp(x))$ \\
$\displaystyle || \vx ||_p $ & $\vx$ 的 $\normlp$ 范数\\
$\displaystyle || \vx || $ & $\vx$ 的 $\normltwo$ 范数\\
$\displaystyle x^+$ & $x$ 的正值部分, 即 $\max(0,x)$\\
$\displaystyle \1_\mathrm{condition}$ & 如果条件为真则值为1,条件为假则值为0 \\
\end{tabular}
\egroup
\index{Sigmoid}
\index{Softplus}
\index{Norm}
\end{minipage}

% Sometimes we use a function $f$ whose argument is a scalar but apply
% it to a vector, matrix, or tensor: $f(\vx)$, $f(\mX)$, or $f(\tX)$.
% This denotes the application of $f$ to the
% array element-wise. For example, if $\tC = \sigma(\tX)$, then $\etC_{i,j,k} = \sigma(\etX_{i,j,k})$
% for all valid values of $i$, $j$ and $k$.
有时我们把参数为标量的函数 $f$ 应用到矢量,矩阵,或张量中:$f(\vx)$, $f(\mX)$, 或 $f(\tX)$。
这代表将 $f$ 应用到数组元素层面,例如,如果 $\tC = \sigma(\tX)$, 那么对所有 $i$, $j$ 和 $k$ 的有效值,
有 $\etC_{i,j,k} = \sigma(\etX_{i,j,k})$


\vspace{\notationgap}
\begin{minipage}{\textwidth}
% \centerline{\bf Datasets and Distributions}
\centerline{\bf 数据集和分布}
\bgroup
\def\arraystretch{1.5}
\begin{tabular}{cp{3.25in}}
% $\displaystyle \pdata$ & The data generating distribution\\
% $\displaystyle \ptrain$ & The empirical distribution defined by the training set\\
% $\displaystyle \sX$ & A set of training examples\\
% $\displaystyle \vx^{(i)}$ & The $i$-th example (input) from a dataset\\
% $\displaystyle y^{(i)}\text{ or }\vy^{(i)}$ & The target associated with $\vx^{(i)}$ for supervised learning\\
% $\displaystyle \mX$ & The $m \times n$ matrix with input example $\vx^{(i)}$ in row $\mX_{i,:}$\\
$\displaystyle \pdata$ & 数据生成分布\\
$\displaystyle \ptrain$ & 由训练集定义的经验分布\\
$\displaystyle \sX$ & 一组训练用例\\
$\displaystyle \vx^{(i)}$ & 数据集中的第 $i$ 个(输入)用例\\
$\displaystyle y^{(i)}\text{ or }\vy^{(i)}$ & 有监督学习下 $\vx^{(i)}$ 对应的目标值\\
$\displaystyle \mX$ & $m \times n$ 的矩阵,其中输入用例 $\vx^{(i)}$ 在 $\mX_{i,:}$ 行\\
\end{tabular}
\egroup
\end{minipage}

\clearpage

\typeout{END_CHAPTER "notation" \theabspage}
