\newif\ifColor
% Add \Colortrue to settings.tex to use color
% Otherwise, will show a primarily grayscale version of the document for printing







































 
\documentclass[11pt,oneside,a4paper]{book}
\usepackage{natbib}
\usepackage{breakcites} % Do Not let citations break out of the text frame
\usepackage{microtype} % Get rid of some frame busts automatically
% Note: if microtype causes error on ubuntu, run
% sudo apt-get install cm-super

%%%%%%% 设置中文字体 %%%%%%%%%%
\usepackage[UTF8]{ctex}

\title{机器学习的数学笔记}
\author{X} 
\date{}
\setcounter{tocdepth}{1}
  
%\pdfobjcompresslevel=0
 
\usepackage{zref-abspage}

\setcounter{secnumdepth}{3} % Number subsubsections, because we reference them,
% so the reader needs numbers to find the correct place.


\usepackage[vcentering,dvips]{geometry}
\geometry{papersize={7in,9in},bottom=3pc,top=5pc,left=5pc,right=5pc,bmargin=4.5pc,footskip=18pt,headsep=25pt}


%%% Packages %%%
\usepackage{epsfig}
\usepackage{subfigure}
\usepackage[utf8]{inputenc}

 
% Needed for some foreign characters
\usepackage[T1]{fontenc}
\usepackage{bbm}
\usepackage{mathtools}
\usepackage{amsmath}
\usepackage{subfigure}
\usepackage{amsfonts}
\usepackage{amsthm}
\usepackage{multirow}
\usepackage{colortbl}
\usepackage{booktabs}
% This allows us to cite chapters by name, which was useful for making the
% acknowledgements page
\usepackage{nameref}
% Make sure there is a space between the subsection number and subsection title
% in the table of contents.
% If we do not do this we end up with 2 digit subsection numbers colliding with
% the title.
% See https://tex.stackexchange.com/questions/7853/toc-text-numbers-alignment/7856#7856?newreg=d2632892dd0345f388619f12fa794b11
\usepackage[tocindentauto]{tocstyle}
\usetocstyle{standard}

\usepackage{bm}


\usepackage{float}
\newcommand{\boldindex}[1]{\textbf{\hyperpage{#1}}}
\usepackage{makeidx}\makeindex
% Make bibliography and index appear in table of contents
\usepackage[nottoc]{tocbibind}
% Using the caption package allows us to support captions that contain "itemize" environments.
% The font=small option makes the text of captions smaller than the main text.
\usepackage[font=small]{caption}

% Used to change header sizes
\usepackage{fancyhdr}



\usepackage[chapter]{algorithm}
\usepackage{algorithmic}
% Include chapter number in algorithm number
\renewcommand{\thealgorithm}{\arabic{chapter}.\arabic{algorithm}}


\theoremstyle{definition}
\newtheorem{example}{Example}[section]
\newtheorem{definition}{Definition}[section]
\newtheorem{lemma}{Lemma}[section]
\newtheorem{theorem}{Theorem}[section]


% Define the P table cell environment
% It is the same as p, but centers the text horizontally
\usepackage{array}
\newcolumntype{P}[1]{>{\centering\arraybackslash}p{#1}}

% Rebuild the book document class's headers from scratch, but with different font size
% (this is for MIT Press style)
% Source: http://texblog.org/2012/10/09/changing-the-font-size-in-fancyhdr/
\newcommand{\changefont}{% 1st arg to fontsize is font size. 2nd arg is the baseline skip. both in points.
    \fontsize{9}{11}\selectfont
}
\fancyhf{}
%\fancyhead[LE,RO]{\changefont \slshape \rightmark} %section
\fancyhead[RE,LO]{\changefont \slshape \leftmark} %chapter
\fancyfoot[C]{\changefont \thepage} %footer
\pagestyle{fancy}
%%%%% NEW MATH DEFINITIONS %%%%%

% Mark sections of captions for referring to divisions of figures
\newcommand{\figleft}{{\em (Left)}}
\newcommand{\figcenter}{{\em (Center)}}
\newcommand{\figright}{{\em (Right)}}
\newcommand{\figtop}{{\em (Top)}}
\newcommand{\figbottom}{{\em (Bottom)}}
\newcommand{\captiona}{{\em (a)}}
\newcommand{\captionb}{{\em (b)}}
\newcommand{\captionc}{{\em (c)}}
\newcommand{\captiond}{{\em (d)}}

% Highlight a newly defined term
\newcommand{\newterm}[1]{{\bf #1}}


% Figure reference, lower-case.
\def\figref#1{图~\ref{#1}}
% Figure reference, capital. For start of sentence
\def\Figref#1{图~\ref{#1}}
\def\twofigref#1#2{图 \ref{#1} 和 \ref{#2}}
\def\quadfigref#1#2#3#4{图 \ref{#1}, \ref{#2}, \ref{#3} 以及 \ref{#4}}
% Section reference, lower-case.
\def\secref#1{第~\ref{#1}节}
% Section reference, capital.
\def\Secref#1{第~\ref{#1}节}
% Reference to two sections.
\def\twosecrefs#1#2{第 \ref{#1} 节和 \ref{#2}节}
% Reference to three sections.
\def\secrefs#1#2#3{第 \ref{#1}节, 第\ref{#2}节和第 \ref{#3}节}
% Reference to an equation, lower-case. % 公式加了括号
\def\eqref#1{公式~(\ref{#1})}
% Reference to an equation, upper case
\def\Eqref#1{公式~(\ref{#1})}
% A raw reference to an equation---avoid using if possible
\def\plaineqref#1{\ref{#1}}
% Reference to a chapter, lower-case.
\def\chapref#1{第~\ref{#1}章}
% Reference to an equation, upper case.
\def\Chapref#1{第~\ref{#1}章}
% Reference to a range of chapters
\def\rangechapref#1#2{第\ref{#1}--\ref{#2}章}
% Reference to an algorithm, lower-case.
\def\algref#1{算法~\ref{#1}}
% Reference to an algorithm, upper case.
\def\Algref#1{算法~\ref{#1}}
\def\twoalgref#1#2{算法 \ref{#1} 和 \ref{#2}}
\def\Twoalgref#1#2{算法 \ref{#1} 和 \ref{#2}}
% Reference to a part, lower case
\def\partref#1{第~\ref{#1}段}
% Reference to a part, upper case
\def\Partref#1{第~\ref{#1}段}
\def\twopartref#1#2{第 \ref{#1} 段和第 \ref{#2}段}

\def\ceil#1{\lceil #1 \rceil}
\def\floor#1{\lfloor #1 \rfloor}
\def\1{\bm{1}}
\newcommand{\train}{\mathcal{D}}
\newcommand{\valid}{\mathcal{D_{\mathrm{valid}}}}
\newcommand{\test}{\mathcal{D_{\mathrm{test}}}}

\def\eps{{\epsilon}}


% Random variables
\def\reta{{\textnormal{$\eta$}}}
\def\ra{{\textnormal{a}}}
\def\rb{{\textnormal{b}}}
\def\rc{{\textnormal{c}}}
\def\rd{{\textnormal{d}}}
\def\re{{\textnormal{e}}}
\def\rf{{\textnormal{f}}}
\def\rg{{\textnormal{g}}}
\def\rh{{\textnormal{h}}}
\def\ri{{\textnormal{i}}}
\def\rj{{\textnormal{j}}}
\def\rk{{\textnormal{k}}}
\def\rl{{\textnormal{l}}}
% rm is already a command, just don't name any random variables m
\def\rn{{\textnormal{n}}}
\def\ro{{\textnormal{o}}}
\def\rp{{\textnormal{p}}}
\def\rq{{\textnormal{q}}}
\def\rr{{\textnormal{r}}}
\def\rs{{\textnormal{s}}}
\def\rt{{\textnormal{t}}}
\def\ru{{\textnormal{u}}}
\def\rv{{\textnormal{v}}}
\def\rw{{\textnormal{w}}}
\def\rx{{\textnormal{x}}}
\def\ry{{\textnormal{y}}}
\def\rz{{\textnormal{z}}}

% Random vectors
\def\rvepsilon{{\mathbf{\epsilon}}}
\def\rvtheta{{\mathbf{\theta}}}
\def\rva{{\mathbf{a}}}
\def\rvb{{\mathbf{b}}}
\def\rvc{{\mathbf{c}}}
\def\rvd{{\mathbf{d}}}
\def\rve{{\mathbf{e}}}
\def\rvf{{\mathbf{f}}}
\def\rvg{{\mathbf{g}}}
\def\rvh{{\mathbf{h}}}
\def\rvu{{\mathbf{i}}}
\def\rvj{{\mathbf{j}}}
\def\rvk{{\mathbf{k}}}
\def\rvl{{\mathbf{l}}}
\def\rvm{{\mathbf{m}}}
\def\rvn{{\mathbf{n}}}
\def\rvo{{\mathbf{o}}}
\def\rvp{{\mathbf{p}}}
\def\rvq{{\mathbf{q}}}
\def\rvr{{\mathbf{r}}}
\def\rvs{{\mathbf{s}}}
\def\rvt{{\mathbf{t}}}
\def\rvu{{\mathbf{u}}}
\def\rvv{{\mathbf{v}}}
\def\rvw{{\mathbf{w}}}
\def\rvx{{\mathbf{x}}}
\def\rvy{{\mathbf{y}}}
\def\rvz{{\mathbf{z}}}

% Elements of random vectors
\def\erva{{\textnormal{a}}}
\def\ervb{{\textnormal{b}}}
\def\ervc{{\textnormal{c}}}
\def\ervd{{\textnormal{d}}}
\def\erve{{\textnormal{e}}}
\def\ervf{{\textnormal{f}}}
\def\ervg{{\textnormal{g}}}
\def\ervh{{\textnormal{h}}}
\def\ervi{{\textnormal{i}}}
\def\ervj{{\textnormal{j}}}
\def\ervk{{\textnormal{k}}}
\def\ervl{{\textnormal{l}}}
\def\ervm{{\textnormal{m}}}
\def\ervn{{\textnormal{n}}}
\def\ervo{{\textnormal{o}}}
\def\ervp{{\textnormal{p}}}
\def\ervq{{\textnormal{q}}}
\def\ervr{{\textnormal{r}}}
\def\ervs{{\textnormal{s}}}
\def\ervt{{\textnormal{t}}}
\def\ervu{{\textnormal{u}}}
\def\ervv{{\textnormal{v}}}
\def\ervw{{\textnormal{w}}}
\def\ervx{{\textnormal{x}}}
\def\ervy{{\textnormal{y}}}
\def\ervz{{\textnormal{z}}}

% Random matrices
\def\rmA{{\mathbf{A}}}
\def\rmB{{\mathbf{B}}}
\def\rmC{{\mathbf{C}}}
\def\rmD{{\mathbf{D}}}
\def\rmE{{\mathbf{E}}}
\def\rmF{{\mathbf{F}}}
\def\rmG{{\mathbf{G}}}
\def\rmH{{\mathbf{H}}}
\def\rmI{{\mathbf{I}}}
\def\rmJ{{\mathbf{J}}}
\def\rmK{{\mathbf{K}}}
\def\rmL{{\mathbf{L}}}
\def\rmM{{\mathbf{M}}}
\def\rmN{{\mathbf{N}}}
\def\rmO{{\mathbf{O}}}
\def\rmP{{\mathbf{P}}}
\def\rmQ{{\mathbf{Q}}}
\def\rmR{{\mathbf{R}}}
\def\rmS{{\mathbf{S}}}
\def\rmT{{\mathbf{T}}}
\def\rmU{{\mathbf{U}}}
\def\rmV{{\mathbf{V}}}
\def\rmW{{\mathbf{W}}}
\def\rmX{{\mathbf{X}}}
\def\rmY{{\mathbf{Y}}}
\def\rmZ{{\mathbf{Z}}}

% Elements of random matrices
\def\ermA{{\textnormal{A}}}
\def\ermB{{\textnormal{B}}}
\def\ermC{{\textnormal{C}}}
\def\ermD{{\textnormal{D}}}
\def\ermE{{\textnormal{E}}}
\def\ermF{{\textnormal{F}}}
\def\ermG{{\textnormal{G}}}
\def\ermH{{\textnormal{H}}}
\def\ermI{{\textnormal{I}}}
\def\ermJ{{\textnormal{J}}}
\def\ermK{{\textnormal{K}}}
\def\ermL{{\textnormal{L}}}
\def\ermM{{\textnormal{M}}}
\def\ermN{{\textnormal{N}}}
\def\ermO{{\textnormal{O}}}
\def\ermP{{\textnormal{P}}}
\def\ermQ{{\textnormal{Q}}}
\def\ermR{{\textnormal{R}}}
\def\ermS{{\textnormal{S}}}
\def\ermT{{\textnormal{T}}}
\def\ermU{{\textnormal{U}}}
\def\ermV{{\textnormal{V}}}
\def\ermW{{\textnormal{W}}}
\def\ermX{{\textnormal{X}}}
\def\ermY{{\textnormal{Y}}}
\def\ermZ{{\textnormal{Z}}}

% Vectors
\def\vzero{{\bm{0}}}
\def\vone{{\bm{1}}}
\def\vmu{{\bm{\mu}}}
\def\vtheta{{\bm{\theta}}}
\def\va{{\bm{a}}}
\def\vb{{\bm{b}}}
\def\vc{{\bm{c}}}
\def\vd{{\bm{d}}}
\def\ve{{\bm{e}}}
\def\vf{{\bm{f}}}
\def\vg{{\bm{g}}}
\def\vh{{\bm{h}}}
\def\vi{{\bm{i}}}
\def\vj{{\bm{j}}}
\def\vk{{\bm{k}}}
\def\vl{{\bm{l}}}
\def\vm{{\bm{m}}}
\def\vn{{\bm{n}}}
\def\vo{{\bm{o}}}
\def\vp{{\bm{p}}}
\def\vq{{\bm{q}}}
\def\vr{{\bm{r}}}
\def\vs{{\bm{s}}}
\def\vt{{\bm{t}}}
\def\vu{{\bm{u}}}
\def\vv{{\bm{v}}}
\def\vw{{\bm{w}}}
\def\vx{{\bm{x}}}
\def\vy{{\bm{y}}}
\def\vz{{\bm{z}}}

% Elements of vectors
\def\evalpha{{\alpha}}
\def\evbeta{{\beta}}
\def\evepsilon{{\epsilon}}
\def\evlambda{{\lambda}}
\def\evomega{{\omega}}
\def\evmu{{\mu}}
\def\evpsi{{\psi}}
\def\evsigma{{\sigma}}
\def\evtheta{{\theta}}
\def\eva{{a}}
\def\evb{{b}}
\def\evc{{c}}
\def\evd{{d}}
\def\eve{{e}}
\def\evf{{f}}
\def\evg{{g}}
\def\evh{{h}}
\def\evi{{i}}
\def\evj{{j}}
\def\evk{{k}}
\def\evl{{l}}
\def\evm{{m}}
\def\evn{{n}}
\def\evo{{o}}
\def\evp{{p}}
\def\evq{{q}}
\def\evr{{r}}
\def\evs{{s}}
\def\evt{{t}}
\def\evu{{u}}
\def\evv{{v}}
\def\evw{{w}}
\def\evx{{x}}
\def\evy{{y}}
\def\evz{{z}}

% Matrix
\def\mA{{\bm{A}}}
\def\mB{{\bm{B}}}
\def\mC{{\bm{C}}}
\def\mD{{\bm{D}}}
\def\mE{{\bm{E}}}
\def\mF{{\bm{F}}}
\def\mG{{\bm{G}}}
\def\mH{{\bm{H}}}
\def\mI{{\bm{I}}}
\def\mJ{{\bm{J}}}
\def\mK{{\bm{K}}}
\def\mL{{\bm{L}}}
\def\mM{{\bm{M}}}
\def\mN{{\bm{N}}}
\def\mO{{\bm{O}}}
\def\mP{{\bm{P}}}
\def\mQ{{\bm{Q}}}
\def\mR{{\bm{R}}}
\def\mS{{\bm{S}}}
\def\mT{{\bm{T}}}
\def\mU{{\bm{U}}}
\def\mV{{\bm{V}}}
\def\mW{{\bm{W}}}
\def\mX{{\bm{X}}}
\def\mY{{\bm{Y}}}
\def\mZ{{\bm{Z}}}
\def\mBeta{{\bm{\beta}}}
\def\mPhi{{\bm{\Phi}}}
\def\mLambda{{\bm{\Lambda}}}
\def\mSigma{{\bm{\Sigma}}}

% Tensor
\DeclareMathAlphabet{\mathsfit}{\encodingdefault}{\sfdefault}{m}{sl}
\SetMathAlphabet{\mathsfit}{bold}{\encodingdefault}{\sfdefault}{bx}{n}
\newcommand{\tens}[1]{\bm{\mathsfit{#1}}}
\def\tA{{\tens{A}}}
\def\tB{{\tens{B}}}
\def\tC{{\tens{C}}}
\def\tD{{\tens{D}}}
\def\tE{{\tens{E}}}
\def\tF{{\tens{F}}}
\def\tG{{\tens{G}}}
\def\tH{{\tens{H}}}
\def\tI{{\tens{I}}}
\def\tJ{{\tens{J}}}
\def\tK{{\tens{K}}}
\def\tL{{\tens{L}}}
\def\tM{{\tens{M}}}
\def\tN{{\tens{N}}}
\def\tO{{\tens{O}}}
\def\tP{{\tens{P}}}
\def\tQ{{\tens{Q}}}
\def\tR{{\tens{R}}}
\def\tS{{\tens{S}}}
\def\tT{{\tens{T}}}
\def\tU{{\tens{U}}}
\def\tV{{\tens{V}}}
\def\tW{{\tens{W}}}
\def\tX{{\tens{X}}}
\def\tY{{\tens{Y}}}
\def\tZ{{\tens{Z}}}


% Graph
\def\gA{{\mathcal{A}}}
\def\gB{{\mathcal{B}}}
\def\gC{{\mathcal{C}}}
\def\gD{{\mathcal{D}}}
\def\gE{{\mathcal{E}}}
\def\gF{{\mathcal{F}}}
\def\gG{{\mathcal{G}}}
\def\gH{{\mathcal{H}}}
\def\gI{{\mathcal{I}}}
\def\gJ{{\mathcal{J}}}
\def\gK{{\mathcal{K}}}
\def\gL{{\mathcal{L}}}
\def\gM{{\mathcal{M}}}
\def\gN{{\mathcal{N}}}
\def\gO{{\mathcal{O}}}
\def\gP{{\mathcal{P}}}
\def\gQ{{\mathcal{Q}}}
\def\gR{{\mathcal{R}}}
\def\gS{{\mathcal{S}}}
\def\gT{{\mathcal{T}}}
\def\gU{{\mathcal{U}}}
\def\gV{{\mathcal{V}}}
\def\gW{{\mathcal{W}}}
\def\gX{{\mathcal{X}}}
\def\gY{{\mathcal{Y}}}
\def\gZ{{\mathcal{Z}}}

% Sets
\def\sA{{\mathbb{A}}}
\def\sB{{\mathbb{B}}}
\def\sC{{\mathbb{C}}}
\def\sD{{\mathbb{D}}}
% Don't use a set called E, because this would be the same as our symbol
% for expectation.
\def\sF{{\mathbb{F}}}
\def\sG{{\mathbb{G}}}
\def\sH{{\mathbb{H}}}
\def\sI{{\mathbb{I}}}
\def\sJ{{\mathbb{J}}}
\def\sK{{\mathbb{K}}}
\def\sL{{\mathbb{L}}}
\def\sM{{\mathbb{M}}}
\def\sN{{\mathbb{N}}}
\def\sO{{\mathbb{O}}}
\def\sP{{\mathbb{P}}}
\def\sQ{{\mathbb{Q}}}
\def\sR{{\mathbb{R}}}
\def\sS{{\mathbb{S}}}
\def\sT{{\mathbb{T}}}
\def\sU{{\mathbb{U}}}
\def\sV{{\mathbb{V}}}
\def\sW{{\mathbb{W}}}
\def\sX{{\mathbb{X}}}
\def\sY{{\mathbb{Y}}}
\def\sZ{{\mathbb{Z}}}

% Entries of a matrix
\def\emLambda{{\Lambda}}
\def\emA{{A}}
\def\emB{{B}}
\def\emC{{C}}
\def\emD{{D}}
\def\emE{{E}}
\def\emF{{F}}
\def\emG{{G}}
\def\emH{{H}}
\def\emI{{I}}
\def\emJ{{J}}
\def\emK{{K}}
\def\emL{{L}}
\def\emM{{M}}
\def\emN{{N}}
\def\emO{{O}}
\def\emP{{P}}
\def\emQ{{Q}}
\def\emR{{R}}
\def\emS{{S}}
\def\emT{{T}}
\def\emU{{U}}
\def\emV{{V}}
\def\emW{{W}}
\def\emX{{X}}
\def\emY{{Y}}
\def\emZ{{Z}}
\def\emSigma{{\Sigma}}

% entries of a tensor
% Same font as tensor, without \bm wrapper
\newcommand{\etens}[1]{\mathsfit{#1}}
\def\etLambda{{\etens{\Lambda}}}
\def\etA{{\etens{A}}}
\def\etB{{\etens{B}}}
\def\etC{{\etens{C}}}
\def\etD{{\etens{D}}}
\def\etE{{\etens{E}}}
\def\etF{{\etens{F}}}
\def\etG{{\etens{G}}}
\def\etH{{\etens{H}}}
\def\etI{{\etens{I}}}
\def\etJ{{\etens{J}}}
\def\etK{{\etens{K}}}
\def\etL{{\etens{L}}}
\def\etM{{\etens{M}}}
\def\etN{{\etens{N}}}
\def\etO{{\etens{O}}}
\def\etP{{\etens{P}}}
\def\etQ{{\etens{Q}}}
\def\etR{{\etens{R}}}
\def\etS{{\etens{S}}}
\def\etT{{\etens{T}}}
\def\etU{{\etens{U}}}
\def\etV{{\etens{V}}}
\def\etW{{\etens{W}}}
\def\etX{{\etens{X}}}
\def\etY{{\etens{Y}}}
\def\etZ{{\etens{Z}}}

% The true underlying data generating distribution
\newcommand{\pdata}{p_{\rm{data}}}
% The empirical distribution defined by the training set
\newcommand{\ptrain}{\hat{p}_{\rm{data}}}
\newcommand{\Ptrain}{\hat{P}_{\rm{data}}}
% The model distribution
\newcommand{\pmodel}{p_{\rm{model}}}
\newcommand{\Pmodel}{P_{\rm{model}}}
\newcommand{\ptildemodel}{\tilde{p}_{\rm{model}}}
% Stochastic autoencoder distributions
\newcommand{\pencode}{p_{\rm{encoder}}}
\newcommand{\pdecode}{p_{\rm{decoder}}}
\newcommand{\precons}{p_{\rm{reconstruct}}}

\newcommand{\laplace}{\mathrm{Laplace}} % Laplace distribution

\newcommand{\E}{\mathbb{E}}
\newcommand{\Ls}{\mathcal{L}}
\newcommand{\R}{\mathbb{R}}
\newcommand{\emp}{\tilde{p}}
\newcommand{\lr}{\alpha}
\newcommand{\reg}{\lambda}
\newcommand{\rect}{\mathrm{rectifier}}
\newcommand{\softmax}{\mathrm{softmax}}
\newcommand{\sigmoid}{\sigma}
\newcommand{\softplus}{\zeta}
\newcommand{\KL}{D_{\mathrm{KL}}}
\newcommand{\Var}{\mathrm{Var}}
\newcommand{\standarderror}{\mathrm{SE}}
\newcommand{\Cov}{\mathrm{Cov}}
% Wolfram Mathworld says $L^2$ is for function spaces and $\ell^2$ is for vectors
% But then they seem to use $L^2$ for vectors throughout the site, and so does
% wikipedia.
\newcommand{\normlzero}{L^0}
\newcommand{\normlone}{L^1}
\newcommand{\normltwo}{L^2}
\newcommand{\normlp}{L^p}
\newcommand{\normmax}{L^\infty}

\newcommand{\parents}{Pa} % See usage in notation.tex. Chosen to match Daphne's book.

\DeclareMathOperator*{\argmax}{arg\,max}
\DeclareMathOperator*{\argmin}{arg\,min}

\DeclareMathOperator{\sign}{sign}
\DeclareMathOperator{\Tr}{Tr}
\let\ab\allowbreak


\usepackage[pdffitwindow=false,
pdfview=FitH,
pdfstartview=FitH,
pagebackref=true,
breaklinks=true,
\ifColor
colorlinks,
\fi
bookmarks=false,
plainpages=false]{hyperref}

% Make \[ \] math have equation numbers - 添加了aligned, 方便公式多行对齐的操作
\DeclareRobustCommand{\[}{\begin{equation}\begin{aligned}}
\DeclareRobustCommand{\]}{\end{aligned}\end{equation}}

% Allow align environments to cross page boundaries.
% If we do not do this, we get weird gaps of several inches of white space
% before or after some long align environments.
\allowdisplaybreaks

\begin{document}
%%%%%% 缩小了字号和行间距,使得文字看起来更紧凑  %%%%%%%%
\fontsize{10}{11}
\selectfont  % change font size for chinese character

\setlength{\parskip}{0.25 \baselineskip}
\newlength{\figwidth}
\setlength{\figwidth}{26pc}
% Spacing between notation sections
\newlength{\notationgap}
\setlength{\notationgap}{1pc}

\typeout{START_CHAPTER "TOC" \theabspage}
\frontmatter

 
\maketitle
\tableofcontents
\typeout{END_CHAPTER "TOC" \theabspage}

 \chapter*{Notation}
\label{notation}


\typeout{START_CHAPTER "notation" \theabspage}

% Sometimes we have to include the following line to get this section
% included in the Table of Contents despite being a chapter*
\addcontentsline{toc}{chapter}{Notation}
This section provides a concise reference describing notation used throughout this
document.
If you are unfamiliar with any of the corresponding mathematical concepts,
\citet{dlbook} describe most of these ideas in
chapters 2--4.

\vspace{\notationgap}
% Need to use minipage to keep title of table on same page as table
\begin{minipage}{\textwidth}
% This is a hack to put a little title over the table
% We cannot use "\section*", etc., they appear in the table of contents.
% tocdepth does not work on this chapter.
\centerline{\bf Numbers and Arrays}
\bgroup
% The \arraystretch definition here increases the space between rows in the table,
% so that \displaystyle math has more vertical space.
\def\arraystretch{1.5}
\begin{tabular}{cp{3.25in}}
$\displaystyle a$ & A scalar (integer or real)\\
$\displaystyle \va$ & A vector\\
$\displaystyle \mA$ & A matrix\\
$\displaystyle \tA$ & A tensor\\
$\displaystyle \mI_n$ & Identity matrix with $n$ rows and $n$ columns\\
$\displaystyle \mI$ & Identity matrix with dimensionality implied by context\\
$\displaystyle \ve^{(i)}$ & Standard basis vector $[0,\dots,0,1,0,\dots,0]$ with a 1 at position $i$\\
$\displaystyle \text{diag}(\va)$ & A square, diagonal matrix with diagonal entries given by $\va$\\
$\displaystyle \ra$ & A scalar random variable\\
$\displaystyle \rva$ & A vector-valued random variable\\
$\displaystyle \rmA$ & A matrix-valued random variable\\
\end{tabular}
\egroup
\index{Scalar}
\index{Vector}
\index{Matrix}
\index{Tensor}
\end{minipage}

\vspace{\notationgap}
\begin{minipage}{\textwidth}
\centerline{\bf Sets and Graphs}
\bgroup
\def\arraystretch{1.5}
\begin{tabular}{cp{3.25in}}
$\displaystyle \sA$ & A set\\
$\displaystyle \R$ & The set of real numbers \\
% NOTE: do not use \R^+, because it is ambiguous whether:
% - It includes 0
% - It includes only real numbers, or also infinity.
% We usually do not include infinity, so we may explicitly write
% [0, \infty) to include 0
% (0, \infty) to not include 0
$\displaystyle \{0, 1\}$ & The set containing 0 and 1 \\
$\displaystyle \{0, 1, \dots, n \}$ & The set of all integers between $0$ and $n$\\
$\displaystyle [a, b]$ & The real interval including $a$ and $b$\\
$\displaystyle (a, b]$ & The real interval excluding $a$ but including $b$\\
$\displaystyle \sA \backslash \sB$ & Set subtraction, i.e., the set containing the elements of $\sA$ that are not in $\sB$\\
$\displaystyle \gG$ & A graph\\
$\displaystyle \parents_\gG(\ervx_i)$ & The parents of $\ervx_i$ in $\gG$
\end{tabular}
\egroup
\index{Scalar}
\index{Vector}
\index{Matrix}
\index{Tensor}
\index{Graph}
\index{Set}
\end{minipage}

\vspace{\notationgap}
\begin{minipage}{\textwidth}
\centerline{\bf Indexing}
\bgroup
\def\arraystretch{1.5}
\begin{tabular}{cp{3.25in}}
$\displaystyle \eva_i$ & Element $i$ of vector $\va$, with indexing starting at 1 \\
$\displaystyle \eva_{-i}$ & All elements of vector $\va$ except for element $i$ \\
$\displaystyle \emA_{i,j}$ & Element $i, j$ of matrix $\mA$ \\
$\displaystyle \mA_{i, :}$ & Row $i$ of matrix $\mA$ \\
$\displaystyle \mA_{:, i}$ & Column $i$ of matrix $\mA$ \\
$\displaystyle \etA_{i, j, k}$ & Element $(i, j, k)$ of a 3-D tensor $\tA$\\
$\displaystyle \tA_{:, :, i}$ & 2-D slice of a 3-D tensor\\
$\displaystyle \erva_i$ & Element $i$ of the random vector $\rva$ \\
\end{tabular}
\egroup
\end{minipage}

\vspace{\notationgap}
\begin{minipage}{\textwidth}
\centerline{\bf Linear Algebra Operations}
\bgroup
\def\arraystretch{1.5}
\begin{tabular}{cp{3.25in}}
$\displaystyle \mA^\top$ & Transpose of matrix $\mA$ \\
$\displaystyle \mA^+$ & Moore-Penrose pseudoinverse of $\mA$\\
$\displaystyle \mA \odot \mB $ & Element-wise (Hadamard) product of $\mA$ and $\mB$ \\
% Wikipedia uses \circ for element-wise multiplication but this could be confused with function composition
$\displaystyle \mathrm{det}(\mA)$ & Determinant of $\mA$ \\
\end{tabular}
\egroup
\index{Transpose}
\index{Element-wise product|see {Hadamard product}}
\index{Hadamard product}
\index{Determinant}
\end{minipage}

\vspace{\notationgap}
\begin{minipage}{\textwidth}
\centerline{\bf Calculus}
\bgroup
\def\arraystretch{1.5}
\begin{tabular}{cp{3.25in}}
% NOTE: the [2ex] on the next line adds extra height to that row of the table.
% Without that command, the fraction on the first line is too tall and collides
% with the fraction on the second line.
$\displaystyle\frac{d y} {d x}$ & Derivative of $y$ with respect to $x$\\ [2ex]
$\displaystyle \frac{\partial y} {\partial x} $ & Partial derivative of $y$ with respect to $x$ \\
$\displaystyle \nabla_\vx y $ & Gradient of $y$ with respect to $\vx$ \\
$\displaystyle \nabla_\mX y $ & Matrix derivatives of $y$ with respect to $\mX$ \\
$\displaystyle \nabla_\tX y $ & Tensor containing derivatives of $y$ with respect to $\tX$ \\
$\displaystyle \frac{\partial f}{\partial \vx} $ & Jacobian matrix $\mJ \in \R^{m\times n}$ of $f: \R^n \rightarrow \R^m$\\
$\displaystyle \nabla_\vx^2 f(\vx)\text{ or }\mH( f)(\vx)$ & The Hessian matrix of $f$ at input point $\vx$\\
$\displaystyle \int f(\vx) d\vx $ & Definite integral over the entire domain of $\vx$ \\
$\displaystyle \int_\sS f(\vx) d\vx$ & Definite integral with respect to $\vx$ over the set $\sS$ \\
\end{tabular}
\egroup
\index{Derivative}
\index{Integral}
\index{Jacobian matrix}
\index{Hessian matrix}
\end{minipage}

\vspace{\notationgap}
\begin{minipage}{\textwidth}
\centerline{\bf Probability and Information Theory}
\bgroup
\def\arraystretch{1.5}
\begin{tabular}{cp{3.25in}}
$\displaystyle \ra \bot \rb$ & The random variables $\ra$ and $\rb$ are independent\\
$\displaystyle \ra \bot \rb \mid \rc $ & They are conditionally independent given $\rc$\\
$\displaystyle P(\ra)$ & A probability distribution over a discrete variable\\
$\displaystyle p(\ra)$ & A probability distribution over a continuous variable, or over
a variable whose type has not been specified\\
$\displaystyle \ra \sim P$ & Random variable $\ra$ has distribution $P$\\% so thing on left of \sim should always be a random variable, with name beginning with \r
$\displaystyle  \E_{\rx\sim P} [ f(x) ]\text{ or } \E f(x)$ & Expectation of $f(x)$ with respect to $P(\rx)$ \\
$\displaystyle \Var(f(x)) $ &  Variance of $f(x)$ under $P(\rx)$ \\
$\displaystyle \Cov(f(x),g(x)) $ & Covariance of $f(x)$ and $g(x)$ under $P(\rx)$\\
$\displaystyle H(\rx) $ & Shannon entropy of the random variable $\rx$\\
$\displaystyle \KL ( P \Vert Q ) $ & Kullback-Leibler divergence of P and Q \\
$\displaystyle \mathcal{N} ( \vx ; \vmu , \mSigma)$ & Gaussian distribution %
over $\vx$ with mean $\vmu$ and covariance $\mSigma$ \\
\end{tabular}
\egroup
\index{Independence}
\index{Conditional independence}
\index{Variance}
\index{Covariance}
\index{Kullback-Leibler divergence}
\index{Shannon entropy}
\end{minipage}

\vspace{\notationgap}
\begin{minipage}{\textwidth}
\centerline{\bf Functions}
\bgroup
\def\arraystretch{1.5}
\begin{tabular}{cp{3.25in}}
$\displaystyle f: \sA \rightarrow \sB$ & The function $f$ with domain $\sA$ and range $\sB$\\
$\displaystyle f \circ g $ & Composition of the functions $f$ and $g$ \\
  $\displaystyle f(\vx ; \vtheta) $ & A function of $\vx$ parametrized by $\vtheta$.
  (Sometimes we write $f(\vx)$ and omit the argument $\vtheta$ to lighten notation) \\
$\displaystyle \log x$ & Natural logarithm of $x$ \\
$\displaystyle \sigma(x)$ & Logistic sigmoid, $\displaystyle \frac{1} {1 + \exp(-x)}$ \\
$\displaystyle \zeta(x)$ & Softplus, $\log(1 + \exp(x))$ \\
$\displaystyle || \vx ||_p $ & $\normlp$ norm of $\vx$ \\
$\displaystyle || \vx || $ & $\normltwo$ norm of $\vx$ \\
$\displaystyle x^+$ & Positive part of $x$, i.e., $\max(0,x)$\\
$\displaystyle \1_\mathrm{condition}$ & is 1 if the condition is true, 0 otherwise\\
\end{tabular}
\egroup
\index{Sigmoid}
\index{Softplus}
\index{Norm}
\end{minipage}

Sometimes we use a function $f$ whose argument is a scalar but apply
it to a vector, matrix, or tensor: $f(\vx)$, $f(\mX)$, or $f(\tX)$.
This denotes the application of $f$ to the
array element-wise. For example, if $\tC = \sigma(\tX)$, then $\etC_{i,j,k} = \sigma(\etX_{i,j,k})$
for all valid values of $i$, $j$ and $k$.


\vspace{\notationgap}
\begin{minipage}{\textwidth}
\centerline{\bf Datasets and Distributions}
\bgroup
\def\arraystretch{1.5}
\begin{tabular}{cp{3.25in}}
$\displaystyle \pdata$ & The data generating distribution\\
$\displaystyle \ptrain$ & The empirical distribution defined by the training set\\
$\displaystyle \sX$ & A set of training examples\\
$\displaystyle \vx^{(i)}$ & The $i$-th example (input) from a dataset\\
$\displaystyle y^{(i)}\text{ or }\vy^{(i)}$ & The target associated with $\vx^{(i)}$ for supervised learning\\
$\displaystyle \mX$ & The $m \times n$ matrix with input example $\vx^{(i)}$ in row $\mX_{i,:}$\\
\end{tabular}
\egroup
\end{minipage}

\clearpage

\typeout{END_CHAPTER "notation" \theabspage}

 \mainmatter
 \chapter{方法概论}
\label{chap:methodology}
\typeout{START_CHAPTER "model" \theabspage}

本章讨论\newterm{监督学习}的基本方法与概念,内容参考了\cite{lihang2012StatLearningMethod}和\cite{MITMLMath}。

\section{基本概念}


\subsection{模型}
用$X$和$Y$表示输入空间$\gX$和输出空间$\gY$上的变量,用$\vtheta$表示参数向量,模型的假设空间一般有两种情况
\begin{itemize}
	\item 决策函数的集合$\gF=\{f |Y=f_{\vtheta}(X) \}$,此类模型称为\newterm{非概率模型}
	\item 条件概率的集合$\gF=\{P|P_{\vtheta}(Y|X) \}$,此类模型称为\newterm{概率模型}
\end{itemize}

\subsection{策略}


\subsubsection{损失函数}

度量输出的预测值$f(X)$与真实值$Y$差异程度的函数称为\newterm{损失函数},记做$L(Y,f(X))$,通常的损失函数有0-1损失, 平方损失等。

常用的损失函数有0-1损失,平方损失,对数损失等。

用损失函数的期望来度量模型$f$在联合分布$P(X,Y)$下的平均损失,也成为称为\newterm{风险函数}或\newterm{期望风险}。
\[
	R_{\exp}(f) = \E [ L(Y,f(X)) ] = \int_{\gX \times \gY} L(Y,f(X)) d P(X,Y)
	\nonumber
\]

学习的目标是选择期望最小的模型。然而实践中联合概率分布$P(X,Y)$是未知的,只能用\newterm{经验风险}来估计。

给定一个训练集$\sX=\{(\vx_1, y_1),(\vx_2, y_2)..,(\vx_N,y_N)  \}$
\[
	R_{\operatorname{emp}}(f) = \frac{1}{N} \sum_{i=1}^N L(y_i,f(\vx_i))	
	\nonumber
\]

根据监督学习的基本假设,训练数据和测试数据都是依$P(X,Y)$独立同分布产生的,所以$R_{\operatorname{emp}}(f)$是$R_{\exp}(f)$的无偏估计。根据大数定律,$R_{\operatorname{emp}}(f)$收敛于$R_{\exp}(f)$。

\section{泛化误差}


假设学习到的模型为$\hat{f}$,那么该模型的泛化误差定义为该模型的期望风险:
\[
	R_{\exp}(\hat{f})	=\E [ L(Y,\hat{f}(X)) ] = \int_{\gX \times \gY} L(Y,\hat{f}(X)) d P(X,Y)
\]

下文以二类分类问题为例,讨论模型的泛化误差。

\subsection{二类分类问题}


\clearpage
%%
\typeout{END_CHAPTER "model" \theabspage}

 \chapter{逻辑回归}
\label{chap:logistic}
\typeout{START_CHAPTER "model" \theabspage}


\section{二项逻辑回归模型}
\label{sec:bin_logistic_regression}

二项逻辑回归模型是如下的条件概率分布
\[
	P(Y=1| \vx)=& \frac{\exp(\vtheta^T \vx + b)}{1+\exp(\vtheta^T \vx + b)}\\
	P(Y=0| \vx)=&\frac{1}{1+\exp(\vtheta^T \vx + b)}
\nonumber\]
其中$\vx \in \R^n$是输入变量, $Y \in \{0,1\}$是输出变量,$\vtheta \in \R^n$和$b \in \R$是参数。 $\vx$和$\vtheta$为$n$维列向量。

若令$\vtheta=(\evtheta^{(1)},...,\evtheta^{(n)}, b)^T$, $\vx=(\evx^{(1)},...,\evx^{(n)},1)^T$,那么条件概率可以表示为
\[
	P(Y=1| \vx)=& \frac{\exp(\vtheta^T \vx )}{1+\exp(\vtheta^T \vx)}\\
	P(Y=0| \vx)=&\frac{1}{1+\exp(\vtheta^T \vx )}
	\label{def:bin_logistic}
\]

\subsection{模型的参数估计}
对于给定的训练集$\sX=\{ (\vx_1, y_1),..., (\vx_N, y_N) \}$,可应用极大似然估计法估计模型参数。

为表示方便,令$P(Y=1|\vx)=\pi(\vx), P(Y=0|\vx)=1-\pi(\vx)$, 似然函数为
\[
	L(\vtheta) = \prod^N_{i=1}\left(\pi(\vx_i)\right)^{y_i}\left(1-\pi(\vx_i)\right)^{1-y_i}
	\nonumber 
\]

那么对数似然函数为
\[
	\log L(\vtheta) 
	=& \sum^N_{i=1}\left(y_i\log \pi(\vx_i) + (1-y_i)\log(1-\pi(\vx_i))\right)   \\
	=& \sum^N_{i=1}\left(y_i\log \frac{\pi(\vx_i)}{1-\pi(\vx_i)} + \log(1-\pi(\vx_i))\right)  \\
	=& \sum^N_{i=1}\left(y_i(\vtheta^T \vx_i) -  \log(1+ \exp(\vtheta^T \vx_i))\right)
	\label{eqn:logistic_likelihood}
\]


\subsubsection{参数估计:梯度下降法}
根据\eqref{eqn:logistic_likelihood}, 对数似然函数对$\vtheta$的偏导为
\[
	\nabla_\vtheta \log L(\vtheta) 
	=& \sum^N_{i=1} \left(y_i \vx_i - \frac{\exp(\vtheta^Tx_i) \vx_i}{1+ \exp(\vtheta^T \vx_i)} \right) \\
	=& \sum^N_{i=1} \left(y_i - \pi(x_i) \right)x_i
	\nonumber
\]


由此此处求对数似然函数的最大值,故需要沿着梯度上升的方向进行迭代,迭代公式为
\[ 
	\vtheta 
	\coloneqq& \vtheta + \lr \frac{\partial}{\partial \vtheta}\log L(\vtheta)  \\
	=&  \vtheta + \lr \sum^N_{i=1} \left(y_i - \pi(\vx_i) \right) \vx_i
	\label{eqn:logistic_GD}
\]
其中$\lr$称为学习率,是一个正常数。

\eqref{eqn:logistic_GD}可以用矩阵表示
\[
	\vtheta \coloneqq \vtheta + \lr X^T \mLambda
\]
其中$\mLambda=\left( \begin{array}{c}
	y_1 - \pi(\vx_1)\\
	y_2 - \pi(\vx_2)\\
	...\\
	y_N - \pi(\vx_N)
	  \end{array} \right)_{N \times 1}$,$X$是由训练数据构成的$N \times (n+1)$矩阵(每一行对应一个样本,每一列对应样本的一个维度,其中还包括一维常数项)。

\subsubsection{参数估计:随机梯度下降法}
梯度下降算法在每次更新回归系数时需要遍历整个数据集,当数据集数量庞大或者特征过多时,该方法的计算复杂度太高。改进方法是每次迭代仅用一个样本来更新回归系数,称为\emph{随机梯度下降法}。

具体而言,对于训练集中的每一个样本$(x_i, y_i)$,计算该样本梯度,并依据迭代公式:
\[
	\vtheta \coloneqq \vtheta + \lr  \left(y_i - \pi(\vx_i) \right)\vx_i
	\label{eqn:logistic_SGD}	
\]

与\eqref{eqn:logistic_GD}相比,随机梯度下降的迭代\eqref{eqn:logistic_SGD}中
\begin{itemize}
\item 误差变量是数值,而不是向量
\item 不再有矩阵变换的过程
\end{itemize}

所以随机梯度下降算法的计算效率较高,缺点是存在解的不稳定性(如解存在周期性波动)的问题。为了解决这一问题,并进一步加快收敛速度,可以通过随机选取样本来更新回归系数。


\section{Softmax回归模型}
\label{sec:softmax_regression}

Softmax模型是二项回归模型在多分类问题上的推广,在多分类问题中,类标签$Y$可以取两个以上的值。 

假设$Y$的取值集合是$\{1,2,...,K \}$,Softmax模型是如下的条件概率分布
\[
	P(Y=k| \vx)= \frac{\exp(\vtheta_k^T \vx)}{\xsum{j=1}{K} \exp(\vtheta_j^T \vx)}
\]
其中$\vtheta_1,...,\vtheta_K \in \R^{n+1}$ 是模型的参数。 

为方便起见,下文用矩阵$\mTheta_{K \times (n+1)}$表示全部的模型参数
\[
	\mTheta = \left[   
		\begin{array}{c}
		\vtheta_1^T\\
		. \\
		. \\
		\vtheta_K^T
		\end{array}
			\right]	
	\nonumber			
\]


\subsection{模型的参数估计}
令$P(Y=k|\vx)=\pi_k(\vx)$,与二项逻辑回归类似,Softmax的似然函数可以表示为
\[
	L(\mTheta) = \prod^N_{i=1} \prod^K_{k=1}\prt{\pi_k(\vx_i)}^{\1_\mathrm{y_i=k}}
	\nonumber 
\]

对数似然函数为

\[
	\log L(\mTheta) 
	=& \xsum{i=1}{N} \xsum{k=1}{K} \1_\mathrm{y_i=k} \log \pi_k(\vx_i)    
	\label{eqn:softmax_likelihood}
\]

\subsubsection{参数估计:梯度下降法}
首先求
\[
	\frac{\partial \pi_k(\vx_i)  }{\partial \vtheta_k} = \frac{\vx_i\exp(\vtheta^T_k \vx_i)\prt{\xsum{j=1}{K} \exp(\vtheta_j^T \vx) - \exp(\vtheta^T_k \vx_i)}}{\prt{\xsum{j=1}{K} \exp(\vtheta_j^T \vx)}^2}
\]

故根据\eqref{eqn:softmax_likelihood},得到Softmax模型的对数似然函数的梯度
\[
	\nabla_{\vtheta_k} \log L(\mTheta) 
	=&  \xsum{i=1}{N}  \1_\mathrm{y_i=k}\frac{1}{\pi_k(\vx_i)}\frac{\partial \pi_k(\vx_i)  }{\partial \vtheta_k} \\
	=& \xsum{i=1}{N}  \1_\mathrm{y_i=k}\frac{1}{\pi_k(\vx_i)} \frac{\vx_i\exp(\vtheta^T_k \vx_i)\prt{\xsum{j=1}{K} \exp(\vtheta_j^T \vx_i) - \exp(\vtheta^T_k \vx_i)}}{\prt{\xsum{j=1}{K} \exp(\vtheta_j^T \vx_i)}^2} \\
	=& \xsum{i=1}{N}  \1_\mathrm{y_i=k} \frac{\vx_i\prt{\xsum{j=1}{K} \exp(\vtheta_j^T \vx) - \exp(\vtheta^T_k \vx_i)}}{\xsum{j=1}{K} \exp(\vtheta_j^T \vx)} \\
	=& \xsum{i=1}{N}  \1_\mathrm{y_i=k} \vx_i\prt{1 - \pi_k(\vx_i)  }
\]

对于任意第$k$个分类的参数$\vtheta_k$,可沿着梯度上升的方向进行迭代
\[
   \vtheta_k  \coloneqq \vtheta_k + \lr \xsum{i=1}{N}  \1_\mathrm{y_i=k} \vx_i\prt{1 - \pi_k(\vx_i)  }
   \label{eqn:softmax_gd}
\]

\eqref{eqn:softmax_gd}的迭代关系用矩阵可以表示为
\[
	\vtheta_k  \coloneqq \vtheta_k + \lr X^T \mLambda
\]
其中$\mLambda=\left(
	 \begin{array}{c}
		\1_\mathrm{y_1=k}\prt{1 - \pi_k(\vx_1)}\\
		\1_\mathrm{y_2=k}\prt{1 - \pi_k(\vx_2)}\\
		...\\
		\1_\mathrm{y_N=k}\prt{1 - \pi_k(\vx_N)}
	  \end{array} \right)_{N \times 1}$,$X$是由训练数据构成的$N \times (n+1)$矩阵(每一行对应一个样本,每一列对应样本的一个维度,其中还包括一维常数项)。

\clearpage
%%
\typeout{END_CHAPTER "model" \theabspage}

 \chapter{主成分分析}
\label{chap:pca}
\typeout{START_CHAPTER "model" \theabspage}

主成分分析(Principal Component Analysis, PCA)是一种常见的\newterm{数据降维}方法,其目的是在信息量损失较小的前提下,将高维的数据转换到低维,从而减小计算量。实质就是找到一些投影方向,使得数据在这些投影方向上包含的信息量最大,而且这些投影方向是相互正交的。选择其中一部分包含最多信息量的投影方向作为新的数据空间,同时忽略包含较小信息量的投影方向,从而达到降维的目的。

样本的\newterm{信息量}可以理解为是样本在特征方向上投影的方差。方差越大,则样本在该特征上的差异就越大,因此该特征就越重要。参见《机器学习实战》上的图,在分类问题里,样本的方差越大,越容易将不同类别的样本区分开。

PCA的数学原理,就是对原始的空间中顺序地找一组相互正交的坐标轴,第一个轴是使得方差最大的,第二个轴是在与第一个轴正交的平面中使得方差最大的,第三个轴是在与第1、2个轴正交的平面中方差最大的,这样假设在N维空间中,可以找到N个这样的坐标轴,取前r 个去近似这个空间,这样就从一个N 维的空间压缩到r 维的空间了,但是最终选择的r 个坐标轴能够使得数据的损失最小。


\section{主成分分析的算法}
假设
\begin{itemize}
	\item 存在$n$个原始数据,每个数据有$p$个特征,用矩阵表示为$\mZ_{n \times p}=(\vz_1,\vz_2,...,\vz_n)^T$, 其中$\vz_{i}$ 为$p$维列向量。
\end{itemize}

\begin{enumerate}
	\item 去除平均值,即中心化,将数据\newterm{中心化}变换为$\mX_{n \times p}=(\vx_{1},\vx_{2},...,\vx_{n})^T$, 其中$\mX=\mZ-\E \mZ$(具体而言$\vx_{i}=\vz_{i}-\vmu$, $\vmu=\frac{1}{n}\xsum{i=1}{n}\vz_{i}$)。
	\item 计算$X$的协方差矩阵,用$\mSigma_{p \times p}$ 表示
	\[
		\Var X =& \Var ( Z - \E Z) = \Var Z = \mSigma
		\nonumber
	\]
	实际上$X$的协方差矩阵就是原始数据$Z$的协方差矩阵。
	\item 计算协方差矩阵$\mSigma$的特征向量$\{ \vxi_{j}\}$和特征值$\{\reg_{j}\}$, $j=1..p$。
	\item 将特征值从小到大排序。
	\item 保留前若干个特征值对应的特征向量, 假设保留的特征值为$\{ \reg^*_{j}\} $, $j=1..q$, 对应的特征向量构成的矩阵为$\mXi_{p \times q} =( \vxi^*_{1}, \vxi^*_{2},..., \vxi^*_{q})$
	\item 将数据集$X$转换到上述$q$个特征向量构建的新的空间中, 得到新的数据集$\mX^*_{n \times q} = \mX\mXi= \prt{ \begin{array}{c} \vx_{1} \\\vx_{2}\\... \\ \vx_{i}\\... \\ \vx_{n}   \end{array} } ( \vxi^*_{1},  \vxi^*_{2}, ..., \vxi^*_{q})$
\end{enumerate}


\section{主成分分析的数学原理}
\subsection{几个重要的定理}
\begin{theorem}
	$\mSigma$为对称矩阵,如下优化问题的解$\vu^*$是$\mSigma$的特征向量。
	\[
		\vu^* = \underset{\Vert u\Vert =1}{\argmax} \prt{ \vu^T \mSigma \vu }	
		\nonumber
	\]
	\label{theorem:pca_lemma_ev}
\end{theorem}

\begin{proof}
	实际上约束条件$\Vert \vu \Vert =1$等价于$\vu^T \vu=1$
	
	利用拉格朗日乘子法,得到
	\[
		G(\vu; \reg) = \vu^T \mSigma \vu + \reg (\vu^T \vu-1)	
		\nonumber
	\]
	
	对$G$求$u$的偏导得到
	\[
		\nabla_\vu G(\vu; \reg) 
		=& 2 \mSigma \vu + 2\reg \vu
		\nonumber	
	\]

	如果$\vu^*$是优化问题的解,那么$\vu^*$满足
	\[
		&\nabla_\vu G(\vu; \reg)\mid _{\vu=\vu^*}   = 0 \\
		\Rightarrow&  \mSigma \vu^* = -\reg \vu^* 	
		\nonumber
	\]


	
	所以$\vu^*$是矩阵$\mSigma$的特征向量,对应的特征值为$-\reg$。
	
\end{proof}


\subsection{主成分分析的算法原理}

\subsubsection{最大方差投影}
用$\vu_{p \times 1}$表示某投影方向上的单位向量,那么$\vx_i$在$\vu$ 上的投影可以表示为
\[
	<\vx_i,\vu>=\vx_{i}^{T}\vu
	\nonumber
\]

那么数据集$\mX$在$\vu$ 上的投影向量为$\mY=\mX \vu$,可知$\mY$的均值和方差为
\[
	\E \mY =& \E \mX \vu \\
	\Var \mY =& \Var \mX \vu = \vu^T (\Var \mX) \vu = \vu^T \mSigma \vu
	\nonumber
\]

主成分分析就是要到一个方向,使得数据集$\mX$在该方向上投影方差最大。如果用单位向量$\vu_1$来表示这个方向,那么$\vu_1$是如下优化问题的解
\[
	\underset{\Vert u \Vert=1}{\argmax} \quad \vu^T\mSigma \vu 
	\nonumber
\]

根据\theoref{theorem:pca_lemma_ev},$u_1$就是$\mSigma$的特征向量。





\clearpage
%%
\typeout{END_CHAPTER "model" \theabspage}

 
\chapter{附录:数学基本方法}
\label{app:general}
\typeout{START_CHAPTER "model" \theabspage}

\section{梯度}
本节内容参考了\cite{GradDecent}。

\subsection{梯度与方向导数}
函数在某点的
\begin{itemize}
    \item \newterm{导数}表示函数曲线上的切线斜率,也表示函数在该点的变化率。
    \item \newterm{偏导数}是函数关于其某一个变量的导数,物理含义是函数沿着坐标轴正方向上的变化率。
    \item \newterm{方向导数}是函数在某点沿某个指定方向上的变化率。
    \item \newterm{梯度}是函数在该点沿所有方向变化率最大的那个方向,是一个向量。
\end{itemize}

\subsection{梯度下降}


设函数$u=u(x,y)$在点$P_0(x_0,y_0)$的某空间领域内$U \subset \R$有定义,$l$为从点$P_0$出发的射线,$P(x,y)$为$l$上且在$U$内的任一点, 令
\[
    x - x_0 =& \Delta x = t \cos\lr \\
    y - y_0 =&  \Delta y = t \sin\lr 
    \nonumber
\]

以$t=\sqrt{(\Delta x)^2 + (\Delta y)^2}$ 表示$P$与$P_0$之间的距离,若极限
\[
    \lim_{t \to 0^+} \frac{u(P) - u(P_0)}{t} = \lim_{t \to 0^+} \frac{u(x_0+t \cos\lr, y_0+t \sin\lr) - u(P_0)}{t} \nonumber
\]

存在,则称此极限为函数$u=u(x,y)$在点$P_0(x_0,y_0)$沿着方向$l$的\newterm{方向导数},记做$\frac{\partial u}{\partial l}  \big|_{P_0}$。

\begin{lemma}
假设函数$u=u(x,y)$在点$P_0(x_0,y_0)$可微,则$u=u(x,y)$在点$P_0(x_0,y_0)$沿着方向$l$的方向导数都存在,且
\[
    \frac{\partial u}{\partial l}  \big|_{P_0} =  u^{'}_x(P_0) \cos \lr + u^{'}_y(P_0) \sin \lr  
\]
\end{lemma}

假设函数$u=u(x,y)$在点$P_0(x_0,y_0)$存在一阶偏导数,则定义
\[
    \nabla u \big| _{P_0} = \prt{u^{'}_x(P_0), u^{'}_y(P_0) }  \nonumber
\]
为函数$u=u(x,y)$在点$P_0(x_0,y_0)$的\newterm{梯度}。

\begin{theorem}
    函数$u=u(x,y)$在点$P_0(x_0,y_0)$处的方向导数在其梯度方向上达到最大值,此最大值为梯度的模。
\end{theorem}

\begin{proof}
    根据方向导数的定义
    \[
        \frac{\partial u}{\partial l}  \big|_{P_0} =&  u^{'}_x(P_0) \cos \lr + u^{'}_y(P_0) \sin \lr \nonumber \\
            =& \prt{u^{'}_x(P_0), u^{'}_y(P_0) } \cdot  \prt{ \cos \lr, \sin \lr  } \nonumber \\
            =& \nabla u \big| _{P_0} \cdot l \nonumber \\
            =& \big | \nabla u \big| _{P_0}  \big| \big| l \big| \cos \evtheta \nonumber \\
            =& \big | \nabla u \big| _{P_0}  \big| \cos \evtheta
            \label{eqn:app_general_grad}    
    \]

    其中$\evtheta$ 为梯度向量$\nabla u \big| _{P_0}$与方向向量$l$的夹角。根据公式(\eqref{eqn:app_general_grad})可知,当夹角$\cos \evtheta=1$,即$\evtheta=0$时,
    方向导数$\frac{\partial u}{\partial l}  \big|_{P_0}$取得最大值,此最大值为$\big | \nabla u \big| _{P_0}  \big|$。
\end{proof}





\typeout{END_CHAPTER "model" \theabspage}   
 
\chapter{附录:概率论中的重要结论}
\label{app:entory}
\typeout{START_CHAPTER "model" \theabspage}

\section{Hoeffding不等式}
本节内容参考了\cite{MITMLMath}。

\begin{lemma}[Hoeffding 引理]
    如果随机变量$Z\in [a,b] \quad a.s$,并且$\E Z=0$,那么对于$\forall s \in \R$,都有
    \[
        \E e^{sZ} \le e^{\frac{s^2(b-a)^2}{8}}  
    \]
\end{lemma}

\begin{proof}
    令 $\psi(s)=\log \E e^{sZ}$,由于$Z\in[a,b] \quad a.s$,对于给定的$s$,$e^{sZ}$是有界函数。
    
    根据勒贝格控制收敛定理的结论,可知$e^{sZ}$的积分和求导符号可互换,那么得到
    \[
        \psi^{'}(s)=&  \frac{\E Z e^{sZ}}{\E e^{sZ}} \\
        \psi^{''}(s) =& \frac{\E Z^2 e^{sZ}\E e^{sZ} - (\E Z e^{sZ})^2}{(\E e^{sZ})^2}=\E Z^2 \left( \frac{ e^{sZ}}{\E e^{sZ}}\right) - \left(\E Z \frac{ e^{sZ}}{\E e^{sZ}} \right)^2
        \nonumber
    \]

    因为$\int \frac{ e^{sZ}}{\E e^{sZ}} d \mathbb{P}=1$,可定义Radon-Nikodym导数
    \[
        \frac{d \mathbb{Q} }{d \mathbb{P}} = \frac{ e^{sZ}}{\E e^{sZ}}    
    \]

    由此引入一个新的概率测度$\mathbb{Q}$。$\psi^{''}(s)$可以看做$Z$在概率测度$\mathbb{Q}$ 下的方差
    \[
        \psi^{''}(s) =& \E^\mathbb{Q} Z^2  - \left(\E^\mathbb{Q} Z  \right)^2 \\
        =& \Var (Z)\\
        =&\Var \left(Z - \frac{a+b}{2} \right)  \quad \prt{\textrm{随机变量加减常数不影响方差} }\\
        \le& \E \left(Z - \frac{a+b}{2} \right)^2  \quad \prt{\textrm{根据方差的定义} } \\
        \le& \E\left( \frac{b-a}{2} \right)^2  \quad \prt{\textrm{根据$Z\in[a,b]$得到$|Z-\frac{a+b}{2}| \le \frac{b-a}{2}$}}\\
        =& \frac{(b-a)^2}{4}
        \nonumber
    \]

    所以
    \[
        & \psi(s) = \int^s_0 \int^y_0 \psi^{''}(x) dx dy \le \int^s_0 \int^y_0 \frac{(b-a)^2}{4} dx dy = \frac{s^2(b-a)}{8}  \\
        \Rightarrow& \quad \log \E e^{sZ} \le  \frac{s^2(b-a)}{8}  \\
        \Rightarrow& \quad \E e^{sZ} \le e^{\frac{s^2(b-a)^2}{8}}
        \nonumber
    \]

    

\end{proof}



\begin{theorem}[Hoeffding定理]
    假设$X_1,...X_n$为$n$个独立的随机变量,并且$X \in[0,1] \quad a.s$, 那么对于 $\forall s >0$,都有
    \[
        & P \prt{\frac{1}{n} \sum^n_{i=1}\prt{X_i - \E X_i}  > t }  \le e^{-2nt^2}  \\
        & P \prt{\frac{1}{n} \sum^n_{i=1}\prt{\E X_i - X_i}  > t }  \le e^{-2nt^2}  \\
        & P \prt{\left| \frac{1}{n} \sum^n_{i=1}\prt{\E X_i - X_i} \right| > t }   \le 2e^{-2nt^2}
        \nonumber
    \]
\end{theorem}


\typeout{END_CHAPTER "model" \theabspage}
 \chapter{附录:信息熵}
\label{app:entory}
\typeout{START_CHAPTER "model" \theabspage}

假设
\footnote{本章参考了信息论与编码(http://www.docin.com/p-957983839-f6.html)和信息论基础(https://wenku.baidu.com/view/5319fed3b9f3f90f76c61b1a.html)}
$X$是一个取有限值的离散随机变量(本文只考虑离散情况), 概率分布为$P(X=x_{i})=p_{i},i=1...n$。 

那么$I(x_{i})=-\log p(x_{i})$称为事件$x_{i}$的\newterm{自信息量}, 随机变量$X$ 的\newterm{熵}定义为$X$ 的自信息量的数学期望,即
\[
    H(X)=\E(I(x_{i})) = - \xsum{i=1}{n} p_{i} \log p_{i}
    \nonumber
\]

熵反映的是随机变量不确定程度的大小:熵的值越大,不确定程度越高。

\section{相关概念}

\subsection{条件熵}

\newterm{条件熵}是指在联合概率空间上熵的条件自信息的数学期望。在已知$X$时,$Y$的条件熵为
\[
    H(Y| X)=\E_{\rx,\ry} I(y_{j}|x_{i}) = - \sum_x \sum_y P(x, y) \log P(y|x)
    \label{eqn:cond_entropy}
\]



\begin{lemma}
    与\eqref{eqn:cond_entropy}等价的定义为给定$X$条件下$Y$的条件分布概率的熵的数学期望
    \[
        H(Y| X)=\E_{\rx} H(Y|X=x)=\sum_x P(x)H(Y|X=x)
        \nonumber
    \]
\end{lemma}

\begin{proof}
\[
    H(Y| X)
    =&  - \sum_x \sum_y P(x, y) \log P(y|x)\\
    =& -\sum_x \sum_y P(x)P(y | x) \log P(y | x) \\
    =& -\sum_x P(x)\sum_y P(y | x)\log P(y | x) \quad \prt{\textrm{$P(x)$与$y$无关}}\\
    =& \sum_x P(x)[-\sum_y P(y | x) \log P(y | x)]  \\
    =& \sum_x P(x) H(Y| X=x)  
    \nonumber
\]    
\end{proof}


$H(Y| X)$的含义是已知在$X$发生的前提下,$Y$发生\newterm{新带来的熵}。

\subsection{相对熵}

\newterm{相对熵},也称\newterm{KL散度},交叉熵等,定义为两个概率分布之比的数学期望。

设$Q(x),P(x)$是随机变量$X$中取值的两个概率分布,则$P$对$Q$的相对熵是
\[
    \KL ( P \Vert Q )= \sum_x P(x) \log \frac{P(x)}{Q(x)} = \E_{\rx} \log\frac{P(x)}{Q(x)}    
    \label{eqn:kl_def}
\]

相对熵可以用来度量两个随机变量的“距离”。

\begin{lemma}
    相对熵恒大于等于零。
\end{lemma}
\begin{proof}
    对于任意分布$P,Q$,根据\eqref{eqn:kl_def},可知
    \[
        \KL ( P \Vert Q ) 
        =& \sum_x P(x) \log \frac{P(x)}{Q(x)} \\
        =& -\sum_x P(x) \log \frac{Q(x)}{P(x)} \\
        \ge & -\log(\sum_x P(x) \frac{Q(x)}{P(x)}) \quad \prt{\textrm{对$-logx$应用Jensen不等式} }\\
        =& -\log \sum_x Q(x) \\
        =& -\log1 \\
        =& 0 
        \nonumber
    \]
    
\end{proof}

\subsection{互信息}
两个随机变量$X,Y$的\newterm{互信息},定义为$X,Y$的联合分布和独立分布乘积的相对熵
\[
    I(X,Y) = \KL(P(X,Y) \Vert P(X)P(Y) )
\]

\begin{lemma}
    互信息与条件熵满足如下关系
    \[
        H(X|Y) = H(X) - I(X,Y)
    \]
\end{lemma}
\begin{proof}
    根据\eqref{eqn:kl_def}以及互信息的定义可知
    \[
        I(X,Y) = \sum_{x,y} P(x,y)\log\frac{P(x,y)}{P(x)P(y)}
        \nonumber
    \]

    那么
    \[
        H(X)-I(X,Y) 
        =& -\sum_x P(x)\log P(x) -\sum_{x,y} P(x,y)\log\frac{P(x,y)}{P(x)P(y)} \\
        =& -\sum_x \prt{\sum_y P(x,y)}\log P(x)-\sum_{x,y} P(x,y)\log\frac{P(x,y)}{P(x)P(y)} \\
        =& -\sum_{x,y}P(x,y)\log P(x) - \sum_{x,y}P(x,y)\log\frac{P(x,y)}{P(x)P(y)} \\
        =& -\sum_{x,y}P(x,y)\prt{\log P(x) + \log\frac{P(x,y)}{P(x)P(y)}} \\
        =& -\sum_{x,y}P(x,y)\log\frac{P(x,y)}{P(y)} \\
        =& -\sum_{x,y}P(x,y)\log  P(x \mid y) \\
        =& H(X|Y)    \quad \prt{\textrm{根据\eqref{eqn:cond_entropy}}}
        \nonumber 
    \]
\end{proof}
















 \chapter{附录:贝叶斯决策论}
\label{app:entory}
\typeout{START_CHAPTER "model" \theabspage}

\section{贝叶斯估计}

\subsection{先验概率与后验概率}
\[
    P(\vtheta \vert X) = \frac{P(X \vert \vtheta )P(\vtheta)}{P(X)}
    \label{eqn:app_bayes_eqn}
\]
 
\eqref{eqn:app_bayes_eqn}中$P(\vtheta \vert X)$ 称为\newterm{后验概率},$P(X\vert\vtheta )$称为\newterm{条件概率}(也是似然估计中的\newterm{似然函数}),$P(\vtheta)$称为\newterm{先验概率},$P(X)$是随机变量$X$ 自身的概率,$P(X)$也被称为“证据”(evidence)。

\subsection{参数估计:极大似然、最大后验以及贝叶斯}
基于概率的参数估计的方法主要有三种,分别是\newterm{最大似然估计}(Maximum Likelihood)、\newterm{贝叶斯估计}(Bayes)和\newterm{最大后验估计}(Maximum a posteriori)。

\begin{itemize}
    \item 极大似然估计(ML)把似然函数$L(\vtheta \vert X)$取得最大值时的参数作为估计值。 (对数)极大似然函数的估计参数可以写成
    \[
        \hat{\vtheta}_{ML} = \argmax_{\vtheta} L(\vtheta \vert X) =  \argmax_{\vtheta} \sum_{x \in X} \log P(x\vert\vtheta)   
        \nonumber
    \] 
    
    \item 最大后验估计(MAP)与极大似然估计的区别在于加入了先验概率$P(\vtheta)$,
    \[
        \hat{\vtheta}_{MAP} =& \arg \max_{\vtheta} \frac{P(X\vert\vtheta )P(\vtheta)}{P(X)} \\
        =&  \argmax_{\vtheta} P(X\vert\vtheta )P(\vtheta)\\
        =& \argmax_{\vtheta} \prt{\sum_{x \in X} \log P(x\vert\vtheta) + \log P(\vtheta) }\\
        =& \argmax_{\vtheta} \prt{ L(\vtheta \vert X) + \log P(\vtheta)}
        \nonumber
    \]

    \item 贝叶斯估计不直接估计参数的值,而是允许参数服从一定概率分布,不再是求使得后验概率最大的参数,而是求满足该假设分布的参数的期望作为估计值, 另外还可计算参数的方差,来评估参数估计的准确程度或者置信度。 具体而言就是要计算
    \[
        P(\vtheta |X) = \frac{P(X|\vtheta )P(\vtheta)}{P(X)} 
        = \frac{P(X|\vtheta )P(\vtheta)}{\int_{\evtheta \in \vtheta} P(X \vert\vtheta) P(\vtheta) d\vtheta}
    \]
\end{itemize}

\typeout{END_CHAPTER "model" \theabspage}


 









\small{
\typeout{START_CHAPTER "bib" \theabspage}
\bibliography{notation}
\bibliographystyle{natbib}
\clearpage
\typeout{END_CHAPTER "bib" \theabspage}
}
\typeout{START_CHAPTER "index-" \theabspage}
\printindex
%\clearpage
\typeout{END_CHAPTER "index-" \theabspage}
%\newpage


\end{document}
