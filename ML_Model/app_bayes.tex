\chapter{附录:贝叶斯决策论}
\label{app:entory}
\typeout{START_CHAPTER "model" \theabspage}

\section{贝叶斯估计}

\subsection{先验概率与后验概率}
\[
    P(\vtheta \vert X) = \frac{P(X \vert \vtheta )P(\vtheta)}{P(X)}
    \label{eqn:app_bayes_eqn}
\]
 
\eqref{eqn:app_bayes_eqn}中$P(\vtheta \vert X)$ 称为\newterm{后验概率},$P(X\vert\vtheta )$称为\newterm{条件概率}(也是似然估计中的\newterm{似然函数}),$P(\vtheta)$称为\newterm{先验概率},$P(X)$是随机变量$X$ 自身的概率,$P(X)$也被称为“证据”(evidence)。

\subsection{参数估计:极大似然、最大后验以及贝叶斯}
基于概率的参数估计的方法主要有三种,分别是\newterm{最大似然估计}(Maximum Likelihood)、\newterm{贝叶斯估计}(Bayes)和\newterm{最大后验估计}(Maximum a posteriori)。

\begin{itemize}
    \item 极大似然估计(ML)把似然函数$L(\vtheta \vert X)$取得最大值时的参数作为估计值。 (对数)极大似然函数的估计参数可以写成
    \[
        \hat{\vtheta}_{ML} = \argmax_{\vtheta} L(\vtheta \vert X) =  \argmax_{\vtheta} \sum_{x \in X} \log P(x\vert\vtheta)   
        \nonumber
    \] 
    
    \item 最大后验估计(MAP)与极大似然估计的区别在于加入了先验概率$P(\vtheta)$,
    \[
        \hat{\vtheta}_{MAP} =& \arg \max_{\vtheta} \frac{P(X\vert\vtheta )P(\vtheta)}{P(X)} \\
        =&  \argmax_{\vtheta} P(X\vert\vtheta )P(\vtheta)\\
        =& \argmax_{\vtheta} \prt{\sum_{x \in X} \log P(x\vert\vtheta) + \log P(\vtheta) }\\
        =& \argmax_{\vtheta} \prt{ L(\vtheta \vert X) + \log P(\vtheta)}
        \nonumber
    \]

    \item 贝叶斯估计不直接估计参数的值,而是允许参数服从一定概率分布,不再是求使得后验概率最大的参数,而是求满足该假设分布的参数的期望作为估计值, 另外还可计算参数的方差,来评估参数估计的准确程度或者置信度。 具体而言就是要计算
    \[
        P(\vtheta |X) = \frac{P(X|\vtheta )P(\vtheta)}{P(X)} 
        = \frac{P(X|\vtheta )P(\vtheta)}{\int_{\evtheta \in \vtheta} P(X \vert\vtheta) P(\vtheta) d\vtheta}
    \]
\end{itemize}

\typeout{END_CHAPTER "model" \theabspage}


 







