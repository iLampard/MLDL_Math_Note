\chapter{附录:贝叶斯决策论}
\label{app:entory}
\typeout{START_CHAPTER "model" \theabspage}

\section{先验概率与后验概率}
此部分参考了\cite{MLEAndMAP}。

\subsection{概念}
\[
    P(\vtheta \vert X) = \frac{P(X \vert \vtheta )P(\vtheta)}{P(X)}
    \label{eqn:app_bayes_eqn}
\]
 
\eqref{eqn:app_bayes_eqn}中$P(\vtheta \vert X)$ 称为\newterm{后验概率},$P(X\vert\vtheta )$称为\newterm{条件概率}(也是似然估计中的\newterm{似然函数}),$P(\vtheta)$称为\newterm{先验概率},$P(X)$是随机变量$X$ 自身的概率,$P(X)$也被称为“证据”(evidence)。

\subsection{频率学派与贝叶斯学派} 

对于概率的认知有频率学派和贝叶斯学派两种。
\begin{itemize}
    \item  频率学派:模型参数是未知的定值,观测是随机变量。估计的方法是\newterm{极大似然估计}(Maximum Likelihood),不依赖于先验概率。
    \item  贝叶斯学派:模型参数是随机变量,观测是定值。估计的方法是\newterm{最大后验估计}(Maximum a posteriori):根据已有的经验和知识推断一个先验概率, 然后在新证据不断积累的情况下调整这个概率。
\end{itemize}


\subsection{参数估计:极大似然与最大后验}
极大似然估计(ML)与最大后验估计(MAP)的数学方法的区别如下。

\begin{itemize}
    \item 极大似然估计(ML)把似然函数$P(X \vert \vtheta )$取得最大值时的参数作为估计值。 (对数)极大似然函数的估计参数可以写成
    \[
        \hat{\vtheta}_{ML} = \argmax_{\vtheta} P(X \vert \vtheta ) =  \argmax_{\vtheta} \sum_{x \in X} \log P(x\vert\vtheta)   
        \nonumber
    \] 
    
    \item 最大后验估计(MAP)与极大似然估计的区别在于加入了先验概率$P(\vtheta)$,
    \[
        \hat{\vtheta}_{MAP} 
        =& \argmax_{\vtheta} P(\vtheta  \vert X) \\
        =& \argmax_{\vtheta} \frac{P(X\vert\vtheta )P(\vtheta)}{P(X)} \\
        =&  \argmax_{\vtheta} P(X\vert\vtheta )P(\vtheta)\\
        =& \argmax_{\vtheta} \prt{\sum_{x \in X} \log P(x\vert\vtheta) + \log P(\vtheta) }
        \nonumber
    \]
  
\end{itemize}

\section{共轭分布}
在贝叶斯理论中,如果后验概率$P(\vtheta \vert X)$和先验概率$P(\vtheta)$满足同样的分布律, 那么先验概率分布和后验概率分布就叫做\newterm{共轭分布}, 同时先验分布叫做似然函数的\newterm{共轭先验分布}。

\begin{theorem}
    二项分布的共轭先验是Beta分布。
\end{theorem}

\begin{proof}
假设
\begin{itemize}
    \item 先验分布为Beta分布,$\vtheta \sim \mBeta (\evalpha, \evbeta)$,那么$P(\vtheta | \evalpha, \evbeta) = \frac{1}{\mBeta(\evalpha, \evbeta)} \vtheta^{\evalpha-1} (1-\vtheta)^{\evbeta-1}$,其中$ \mBeta(\evalpha, \evbeta)=\int_0^1 \vtheta^{\evalpha-1} (1-\evtheta)^{\evbeta-1} d\vtheta =\frac{G(\evalpha)G(\evbeta)}{G(\evalpha+\evbeta)}$
    \item 似然概率服从二项分布,$P(X=k \vert \vtheta) = C^k_n  \vtheta^k (1-\vtheta)^{n-k}$
\end{itemize}
后验概率
\[
    P(\vtheta|X=k) =& \frac{P(X=k|\vtheta)P(\vtheta)}{P(X=k)} \\
    =& \frac{P(X=k|\vtheta)P(\vtheta)}{\int_0^1 P(\vtheta, X=k) d\vtheta}\\
    =& \frac{P(X=k|\vtheta)P(\vtheta)}{\int_0^1 P(X=k|\vtheta)p(\vtheta) d\vtheta}\\
    =& \frac{C^k_n  \vtheta^k (1-\vtheta)^{n-k}\frac{1}{\evbeta(\evalpha, \evbeta)} \vtheta^{\evalpha-1} (1-\vtheta)^{\evbeta-1}}{\int_0^1 C^k_n  \vtheta^k (1-\vtheta)^{n-k}\frac{1}{\evbeta(\evalpha, \evbeta)} \vtheta^{\evalpha-1} (1-\vtheta)^{\evbeta-1} d\vtheta } \\
    =& \frac{\vtheta^{\evalpha+k-1} (1-\vtheta)^{\evbeta+n-k-1}}{\int_0^1  \vtheta^{\evalpha+k-1} (1-\vtheta)^{\evbeta+n-k-1} d\vtheta  }\\
    =&  \frac{\vtheta^{\evalpha+k-1} (1-\vtheta)^{\evbeta+n-k-1}}{B(\evalpha+k, \evbeta+n-k) }  \quad \left(\text{根据定义$B(\evalpha, \evbeta)=\int_0^1 \vtheta^{\evalpha-1} (1-\vtheta)^{\evbeta-1} d\vtheta$ }\right)  
    \nonumber
\]

后验概率也服从Beta分布$\vtheta|X=k \sim \mBeta (\evalpha+k, \evbeta+n-k)$,故二项分布的共轭先验是Beta分布。

\end{proof} 

\typeout{END_CHAPTER "model" \theabspage}


 







