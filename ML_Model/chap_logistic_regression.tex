\chapter{逻辑回归}
\label{chap:logistic}
\typeout{START_CHAPTER "model" \theabspage}


\section{二项逻辑回归模型}
\label{sec:bin_logistic_regression}

二项逻辑回归模型是如下的条件概率分布
\[
	P(Y=1| \vx)=& \frac{\exp(\vtheta^T \vx + b)}{1+\exp(\vtheta^T \vx + b)}\\
	P(Y=0| \vx)=&\frac{1}{1+\exp(\vtheta^T \vx + b)}
\nonumber\]
其中$\vx \in \R^n$是输入变量, $Y \in \{0,1\}$是输出变量,$\vtheta \in \R^n$和$b \in \R$是参数。 $\vx$和$\vtheta$为$n$维列向量。

若令$\vtheta=(\evtheta^{1},...,\evtheta^{n}, b)^T$, $\vx=(\evx^{1},...,\evx^{n},1)^T$,那么条件概率可以表示为
\[
	P(Y=1| \vx)=& \frac{\exp(\vtheta^T \vx )}{1+\exp(\vtheta^T \vx)}\\
	P(Y=0| \vx)=&\frac{1}{1+\exp(\vtheta^T \vx )}
\label{def:bin_logistic}\]

\subsection{模型的参数估计}
对于给定的训练集$\sX=\{ (\vx_1, y_1),..., (\vx_N, y_N) \}$,可应用极大似然估计法估计模型参数。

为表示方便,令$P(Y=1|\vx)=\pi(\vx), P(Y=0|\vx)=1-\pi(\vx)$, 似然函数为
\[
	L(\vtheta) = \prod^N_{i=1}\left(\pi(\vx_i)\right)^{y_i}\left(1-\pi(\vx_i)\right)^{1-y_i}
	\nonumber 
\]

那么对数似然函数为
\[
	\log L(\vtheta) 
	=& \sum^N_{i=1}\left(y_i\log \pi(\vx_i) + (1-y_i)\log(1-\pi(\vx_i))\right)   \\
	=& \sum^N_{i=1}\left(y_i\log \frac{\pi(\vx_i)}{1-\pi(\vx_i)} + \log(1-\pi(\vx_i))\right)  \\
	=& \sum^N_{i=1}\left(y_i(\vtheta^T \vx_i) -  \log(1+ \exp(\vtheta^T \vx_i))\right)
\label{eqn:logistic_likelihood}
\]


\subsubsection{参数估计:梯度下降法}
根据公式\eqref{eqn:logistic_likelihood}, 对数似然函数对$\vtheta$的偏导为
\[
	\frac{\partial}{\partial \vtheta}\log L(\vtheta) 
	=& \sum^N_{i=1} \left(y_i \vx_i - \frac{\exp(\vtheta^Tx_i) \vx_i}{1+ \exp(\vtheta^T \vx_i)} \right) \\
	=& \sum^N_{i=1} \left(y_i - \pi(x_i) \right)x_i
	\nonumber
\]


由此此处求对数似然函数的最大值,故需要沿着梯度上升的方向进行迭代,迭代公式为
\[ 
	\vtheta_{t+1} 
	=& \vtheta_t + \lr \frac{\partial}{\partial \vtheta}\log L(\vtheta)  \\
	=&  \vtheta_t + \lr \sum^N_{i=1} \left(y_i - \pi(\vx_i) \right) \vx_i
\label{eqn:logistic_GD}
\]
其中$\lr$称为学习率,是一个正常数。

公式\eqref{eqn:logistic_GD}可以用矩阵表示
\[
	\vtheta_{t+1} = \vtheta_{t} + \lr X^T \mLambda
\]
其中$\mLambda=\left( \begin{array}{c}
	y_1 - \pi(\vx_1)\\
	y_2 - \pi(\vx_2)\\
	...\\
	y_N - \pi(\vx_N)
	  \end{array} \right)_{N \times 1}$,$X$是由训练数据构成的$N \times (n+1)$矩阵(每一行对应一个样本,每一列对应样本的一个维度,其中还包括一维常数项)。

\subsubsection{参数估计:随机梯度下降法}
梯度下降算法在每次更新回归系数时需要遍历整个数据集,当数据集数量庞大或者特征过多时,该方法的计算复杂度太高。改进方法是每次迭代仅用一个样本来更新回归系数,称为\emph{随机梯度下降法}。

具体而言,对于训练集中的每一个样本$(x_i, y_i)$,计算该样本梯度,并依据迭代公式:
\[
	\vtheta_{t+1}=\vtheta_t + \lr  \left(y_i - \pi(\vx_i) \right)\vx_i
\label{eqn:logistic_SGD}	
\]

与公式\eqref{eqn:logistic_GD}相比,随机梯度下降的迭代公式\eqref{eqn:logistic_SGD}中
\begin{itemize}
\item 误差变量是数值,而不是向量
\item 不再有矩阵变换的过程
\end{itemize}

所以随机梯度下降算法的计算效率较高,缺点是存在解的不稳定性(如解存在周期性波动)的问题。为了解决这一问题,并进一步加快收敛速度,可以通过随机选取样本来更新回归系数。

\clearpage
%%
\typeout{END_CHAPTER "model" \theabspage}
