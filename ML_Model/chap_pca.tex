\chapter{主成分分析}
\label{chap:pca}
\typeout{START_CHAPTER "model" \theabspage}

主成分分析(Principal Component Analysis, PCA)是一种常见的\newterm{数据降维}方法,其目的是在信息量损失较小的前提下,将高维的数据转换到低维,从而减小计算量。实质就是找到一些投影方向,使得数据在这些投影方向上包含的信息量最大,而且这些投影方向是相互正交的。选择其中一部分包含最多信息量的投影方向作为新的数据空间,同时忽略包含较小信息量的投影方向,从而达到降维的目的。

样本的\newterm{信息量}可以理解为是样本在特征方向上投影的方差。方差越大,则样本在该特征上的差异就越大,因此该特征就越重要。参见《机器学习实战》上的图,在分类问题里,样本的方差越大,越容易将不同类别的样本区分开。

PCA的数学原理,就是对原始的空间中顺序地找一组相互正交的坐标轴,第一个轴是使得方差最大的,第二个轴是在与第一个轴正交的平面中使得方差最大的,第三个轴是在与第1、2个轴正交的平面中方差最大的,这样假设在N维空间中,可以找到N个这样的坐标轴,取前r 个去近似这个空间,这样就从一个N 维的空间压缩到r 维的空间了,但是最终选择的r 个坐标轴能够使得数据的损失最小。


\section{主成分分析的算法}
假设
\begin{itemize}
	\item 存在$n$个原始数据,每个数据有$p$个特征,用矩阵表示为$\mZ_{n \times p}=(\vz_1,\vz_2,...,\vz_n)^T$, 其中$\vz_{i}$ 为$p$维列向量。
\end{itemize}

\begin{enumerate}
	\item 去除平均值,即中心化,将数据\newterm{中心化}变换为$\mX_{n \times p}=(\vx_{1},\vx_{2},...,\vx_{n})^T$, 其中$\mX=\mZ-\E \mZ$(具体而言$\vx_{i}=\vz_{i}-\vmu$, $\vmu=\frac{1}{n}\xsum{i=1}{n}\vz_{i}$)。
	\item 计算$X$的协方差矩阵,用$\mSigma_{p \times p}$ 表示
	\[
		\Var X =& \Var ( Z - \E Z) = \Var Z = \mSigma
		\nonumber
	\]
	实际上$X$的协方差矩阵就是原始数据$Z$的协方差矩阵。
	\item 计算协方差矩阵$\mSigma$的特征向量$\{ \vxi_{j}\}$和特征值$\{\reg_{j}\}$, $j=1..p$。
	\item 将特征值从小到大排序。
	\item 保留前若干个特征值对应的特征向量, 假设保留的特征值为$\{ \reg^*_{j}\} $, $j=1..q$, 对应的特征向量构成的矩阵为$\mXi_{p \times q} =( \vxi^*_{1}, \vxi^*_{2},..., \vxi^*_{q})$
	\item 将数据集$X$转换到上述$q$个特征向量构建的新的空间中, 得到新的数据集$\mX^*_{n \times q} = \mX\mXi= \prt{ \begin{array}{c} \vx_{1} \\\vx_{2}\\... \\ \vx_{i}\\... \\ \vx_{n}   \end{array} } ( \vxi^*_{1},  \vxi^*_{2}, ..., \vxi^*_{q})$
\end{enumerate}


\section{主成分分析的数学原理}
\subsection{几个重要的定理}
\begin{theorem}
	$\mSigma$为对称矩阵,如下优化问题的解$\vu^*$是$\mSigma$的特征向量。
	\[
		\vu^* = \underset{\Vert u\Vert =1}{\argmax} \prt{ \vu^T \mSigma \vu }	
		\nonumber
	\]
	\label{theorem:pca_lemma_ev}
\end{theorem}

\begin{proof}
	实际上约束条件$\Vert \vu \Vert =1$等价于$\vu^T \vu=1$
	
	利用拉格朗日乘子法,得到
	\[
		G(\vu; \reg) = \vu^T \mSigma \vu + \reg (\vu^T \vu-1)	
		\nonumber
	\]
	
	对$G$求$u$的偏导得到
	\[
		\nabla_\vu G(\vu; \reg) 
		=& 2 \mSigma \vu + 2\reg \vu
		\nonumber	
	\]

	如果$\vu^*$是优化问题的解,那么$\vu^*$满足
	\[
		&\nabla_\vu G(\vu; \reg)\mid _{\vu=\vu^*}   = 0 \\
		\Rightarrow&  \mSigma \vu^* = -\reg \vu^* 	
		\nonumber
	\]


	
	所以$\vu^*$是矩阵$\mSigma$的特征向量,对应的特征值为$-\reg$。
	
\end{proof}


\subsection{主成分分析的算法原理}

\subsubsection{最大方差投影}
用$\vu_{p \times 1}$表示某投影方向上的单位向量,那么$\vx_i$在$\vu$ 上的投影可以表示为
\[
	<\vx_i,\vu>=\vx_{i}^{T}\vu
	\nonumber
\]

那么数据集$\mX$在$\vu$ 上的投影向量为$\mY=\mX \vu$,可知$\mY$的均值和方差为
\[
	\E \mY =& \E \mX \vu \\
	\Var \mY =& \Var \mX \vu = \vu^T (\Var \mX) \vu = \vu^T \mSigma \vu
	\nonumber
\]

主成分分析就是要到一个方向,使得数据集$\mX$在该方向上投影方差最大。如果用单位向量$\vu_1$来表示这个方向,那么$\vu_1$是如下优化问题的解
\[
	\underset{\Vert u \Vert=1}{\argmax} \quad \vu^T\mSigma \vu 
	\nonumber
\]

根据\theoref{theorem:pca_lemma_ev},$u_1$就是$\mSigma$的特征向量。





\clearpage
%%
\typeout{END_CHAPTER "model" \theabspage}
