\chapter{附录:信息熵}
\label{app:entory}
\typeout{START_CHAPTER "model" \theabspage}
本章参考了\citep{infoandcode}和\citep{infobasics}。

假设$X$是一个取有限值的离散随机变量(本文只考虑离散情况), 概率分布为$P$。 

那么$I(X=x_{i})=-\log P(X=x_{i})$称为事件$x_{i}$的\newterm{自信息量}, 随机变量$X$的\newterm{熵}定义为$X$的自信息量的数学期望,即
\[
    H(X)=\E(I(X)) = - \sum_x P(x) \log P(x)
    \nonumber
\]

熵反映的是随机变量不确定程度的大小:熵的值越大,不确定程度越高。

\section{相关概念}

\subsection{条件熵}

\newterm{条件熵}是指在联合概率空间上熵的条件自信息的数学期望。在已知$X$时,$Y$的条件熵为
\[
    H(Y| X)=\E_{\rx,\ry} I(y_{j}|x_{i}) = - \sum_x \sum_y P(x, y) \log P(y|x)
    \label{eqn:app_cond_entropy}
\]



\begin{lemma}
    与\eqref{eqn:app_cond_entropy}等价的定义为给定$X$条件下$Y$的条件分布概率的熵的数学期望
    \[
        H(Y| X)=\E_{\rx} H(Y|X=x)=\sum_x P(x)H(Y|X=x)
        \nonumber
    \]
\end{lemma}

\begin{proof}
\[
    H(Y| X)
    =&  - \sum_x \sum_y P(x, y) \log P(y|x)\\
    =& -\sum_x \sum_y P(x)P(y | x) \log P(y | x) \\
    =& -\sum_x P(x)\sum_y P(y | x)\log P(y | x) \quad \prt{\textrm{$P(x)$与$y$无关}}\\
    =& \sum_x P(x)[-\sum_y P(y | x) \log P(y | x)]  \\
    =& \sum_x P(x) H(Y| X=x)  
    \nonumber
\]    
\end{proof}


$H(Y| X)$的含义是已知在$X$发生的前提下,$Y$发生\newterm{新带来的熵}。

\subsection{相对熵}

\newterm{相对熵},也称\newterm{KL散度},交叉熵等,定义为两个概率分布之比的数学期望。

设$Q(x),P(x)$是随机变量$X$中取值的两个概率分布,则$P$对$Q$的相对熵是
\[
    \KL ( P \Vert Q )= \sum_x P(x) \log \frac{P(x)}{Q(x)} = \E_{\rx} \log\frac{P(x)}{Q(x)}    
    \label{eqn:app_kl_def}
\]

相对熵可以用来度量两个随机变量的“距离”。

\begin{lemma}
    相对熵恒大于等于零。
\end{lemma}
\begin{proof}
    对于任意分布$P,Q$,根据\eqref{eqn:app_kl_def},可知
    \[
        \KL ( P \Vert Q ) 
        =& \sum_x P(x) \log \frac{P(x)}{Q(x)} \\
        =& -\sum_x P(x) \log \frac{Q(x)}{P(x)} \\
        \ge & -\log(\sum_x P(x) \frac{Q(x)}{P(x)}) \quad \prt{\textrm{对$-logx$应用Jensen不等式} }\\
        =& -\log \sum_x Q(x) \\
        =& -\log1 \\
        =& 0 
        \nonumber
    \]
    
\end{proof}

\subsection{互信息}
两个随机变量$X,Y$的\newterm{互信息},定义为$X,Y$的联合分布和独立分布乘积的相对熵
\[
    I(X,Y) = \KL(P(X,Y) \Vert P(X)P(Y) )
\]

\begin{lemma}
    互信息与条件熵满足如下关系
    \[
        H(X|Y) = H(X) - I(X,Y)
    \]
\end{lemma}
\begin{proof}
    根据\eqref{eqn:app_kl_def}以及互信息的定义可知
    \[
        I(X,Y) = \sum_{x,y} P(x,y)\log\frac{P(x,y)}{P(x)P(y)}
        \nonumber
    \]

    那么
    \[
        H(X)-I(X,Y) 
        =& -\sum_x P(x)\log P(x) -\sum_{x,y} P(x,y)\log\frac{P(x,y)}{P(x)P(y)} \\
        =& -\sum_x \prt{\sum_y P(x,y)}\log P(x)-\sum_{x,y} P(x,y)\log\frac{P(x,y)}{P(x)P(y)} \\
        =& -\sum_{x,y}P(x,y)\log P(x) - \sum_{x,y}P(x,y)\log\frac{P(x,y)}{P(x)P(y)} \\
        =& -\sum_{x,y}P(x,y)\prt{\log P(x) + \log\frac{P(x,y)}{P(x)P(y)}} \\
        =& -\sum_{x,y}P(x,y)\log\frac{P(x,y)}{P(y)} \\
        =& -\sum_{x,y}P(x,y)\log  P(x \mid y) \\
        =& H(X|Y)    \quad \prt{\textrm{根据\eqref{eqn:app_cond_entropy}}}
        \nonumber 
    \]
\end{proof}


\section{熵的性质}
 

$X$的熵具有如下几个性质
\begin{itemize}
    \item 非负性:$H(X) \ge 0$。 \footnote{实际上这种非负性对于离散随机变量$X$成立,对连续随机变量$X$不一定成立。这是本文只考虑离散情况的原因。}
    \item 对称性: 当随机变量的概率取值任意互换时,熵不变。
    \[
        H(p_{1},p_{2}...p_{n})=H(p_{2},p_{1}...p_{n})=H(p_{3},p_{1}...p_{n})=..
        \nonumber
    \]  
    \item 可加性: 如果随机变量$X,Y$相互独立,则$H(X,Y)=H(X)+H(Y)$。
    \item 极值性: 对于任意概率分布$P(X=x_{i})=p_{i}$和$P(Y=y_{i})=q_{i}$,$i=1...n$,都有
    \[
        H(X) = - \sum\limits_{i=1}^n p_{i} \log p_{i} \le - \sum\limits_{i=1}^n p_{i} \log q_{i}  
        \label{eqn:app_entropy_max}
    \]
    当$X$和$Y$的概率分布相同时,\eqref{eqn:app_entropy_max}取等号。

    该性质表明,任意概率分布,它对其他概率分布的自信息取数学期望时, 必大于它本身的熵。

    \item 凸性:对于任意概率分布$P(X=x_{i})=p_{i}$和$P(Y=y_{i})=q_{i}$,$i=1...n$,假设随机变量$Z$的分布为$P(Z=z_{i})=\ga_{i}=\lr p_{i} + (1-\lr)q_{i},\lr \in [0,1]$, 那么$Z$的熵满足
    \[
        H(Z) \ge \lr H(X) + (1-\lr) H(Y)
        \label{eqn:app_entropy_convex}
    \]

\end{itemize}


\begin{theorem}[最大熵定理]
    离散随机变量$X$的概率分布为$P(X=x_{i})=p_{i},i=1...n$,那么
    \[
        H(X) \le \log n
    \]
    当$p_{1}=p_{2}=...=\frac{1}{n}$ 时,等号成立。
\end{theorem}
\begin{proof}
    求熵的最大值等价于以下优化问题
    \[
        \max \quad H(x) =& - \sum_{i=1}^n p_{i} \log p_{i}\\
        s.t. \quad \sum_{i=1}^n p_{i}=&1
        \nonumber 
    \]

    利用拉格朗日乘子法构造函数
    \[
        G(p, \reg) = - \sum_{i=1}^n p_{i} \log p_{i} + \reg \prt{\sum_{i=1}^n p_{i} - 1}
        \label{eqn:app_entropy_lag}
    \]  
    
    \eqref{eqn:app_entropy_lag}中分别对$p_{i}$和$\reg$求导,令其为零,得到
    \[
        \frac{\partial G(p, \reg)}{\partial p_{i}} = -\log p_{i} - 1 + \reg =0 \\
        \sum_{i=1}^n p_{i} - 1 = 0        
    \]
     
    由$-\log p_{i} - 1 + \reg =0$可得到 $p_{i}=e^{\reg-1},i=1,2..n$, 由此可知$p_{1}=p_{2}=...=\frac{1}{n}$
    
\end{proof} 


\begin{lemma}[熵的强可加性]
    当随机变量$X,Y$相关的情况下,联合熵满足强可加性,即
    \[
        H(X,Y) =& H(Y) + H(X|Y)\\
        H(X,Y) =& H(X) + H(Y|X)            
    \]
\end{lemma}
\begin{proof}
    \[
        H(Y) + H(X|Y)
        =&- \sum_y P(y) \log P(y) - \sum_x \sum_y P(x, y) \log P(x|y)\\
        =& - \sum_x \sum_y P(x,y) \log P(y) - \sum_x \sum_y P(x, y) \log P(x|y) \\
        =& - \sum_x \sum_y P(x,y)\log P(x,y) \\
        =& H(X,Y)
        \nonumber  
    \]

    同理可证
    \[
        H(X,Y) =& H(X) + H(Y|X)
        \nonumber
    \]
\end{proof}


\begin{lemma}[熵的凸性]
    证明\eqref{eqn:app_entropy_convex}
\end{lemma}
\begin{proof}
    \[
        H(Z)=& - \sum_{i=1}^{n} \ga_{i} \log \ga_{i} \\
        =& - \sum_{i=1}^{n} \lr p_{i} \log \ga_{i}- \sum_{i=1}^{n} (1-\lr) q_{i} \log \ga_{i}\\
        =& - \sum_{i=1}^{n} \lr p_{i} \log \left( \ga_{i} \frac{p_{i}}{p_{i}}\right)- \sum_{i=1}^{n} (1-\lr) q_{i} \log  \left( \ga_{i} \frac{q_{i}}{q_{i}}\right) \\
        =& - \lr \sum_{i=1}^{n} p_{i} \log p_{i} - (1-\lr) \sum_{i=1}^{n} q_{i}\log q_{i} - \lr \sum_{i=1}^{n} p_{i} \log \frac{\ga_{i}}{p_{i}} - (1-\lr)\sum_{i=1}^{n} q_{i} \log \frac{\ga_{i}}{q_{i}} \\
        =& \lr H(X) + (1-\lr) H(Y) - \lr \sum_{i=1}^{n} p_{i} \log \frac{\ga_{i}}{p_{i}} - (1-\lr)\sum_{i=1}^{n} q_{i} \log \frac{\ga_{i}}{q_{i}}  
        \label{app_entorpy_convex_proof}
    \]

    其中\eqref{app_entorpy_convex_proof}的倒数第二项
    \[
        - \lr \sum_{i=1}^{n} p_{i} \log \frac{\ga_{i}}{p_{i}} =& \lr \left(-\sum_{i=1}^{n} p_{i} \log \ga_{i} + \sum_{i=1}^{n} p_{i} \log p_{i}   \right) \\
         \ge & 0  \quad \prt{\text{根据\eqref{eqn:app_entropy_max}}})
         \nonumber
    \]
    
    同理可知\eqref{app_entorpy_convex_proof}的倒数第一项
    \[
        - (1-\lr)\sum_{i=1}^{n} q_{i} \log \frac{\ga_{i}}{q_{i}}  \ge 0    
        \nonumber
    \]
    所以得到
    \[
        H(Z) \ge \lr H(X) + (1-\lr) H(Y)
        \nonumber
    \]

    

\end{proof}


\begin{theorem}
    条件熵小于无条件熵,即$H(X|Y) \le H(X)$
\end{theorem}

\begin{proof}
    \[
        H(X|Y)-H(X) 
        =& - \sum_{x,y} P(x,y) \log P(x|y) + \sum_x P(x) \log P(x) \\
        =& - \sum_{x,y} P(x,y) \log P(x|y) + \sum_{x} \left(\sum_{y}  P(x,y) \right) \log P(x)\\
        =& - \sum_x \sum_y P(y)P(x|y)\log P(x|y) + \sum_{x} \sum_{y}  P(y)P(x|y)  \log P(x)\\
        =& - \sum_y P(y) \prt{\sum_x P(x|y)\log P(x|y) - \sum_{x}  P(x|y)  \log P(x) } \\
        \nonumber
    \]

根据熵的极值性,可知
\[
  -&   \sum_x P(x|y)\log P(x|y) \le -\sum_{x}  P(x|y)  \log P(x) \\
  \Rightarrow& \quad \sum_x P(x|y)\log P(x|y) - \sum_{x}  P(x|y)  \log P(x)  \ge 0 \\
  \Rightarrow& \quad H(X|Y)-H(X) \le 0
  \nonumber
\]

所以

\end{proof}


\subsection{整理得到的公式}

根据本节内容整理得到的重要公式

\begin{itemize}
\item 根据条件熵定义可得
\[
    H(X|Y) = H(X,Y) - H(Y)    
    \label{eqn:app_entropy_ret1}
\]

\item 根据互信息定义展开可得
\[
    H(X|Y) = H(X) - I(X,Y)    
    \label{eqn:app_entropy_ret2}
\]

\item 根据\eqref{eqn:app_entropy_ret1}和\eqref{eqn:app_entropy_ret2}得到的对偶形式
\[
    H(Y|X) =& H(X,Y) - H(X)\\
    H(Y|X) =& H(Y) - I(X,Y)
    \nonumber
\]

\item 多数文献将下式作为互信息的定义公式
\[
    I(X,Y) = H(X) + H(Y) - H(X,Y)
    \nonumber    
\]

\item $H(X|Y) \le H(X)$


\end{itemize}
\typeout{END_CHAPTER "model" \theabspage}


 







